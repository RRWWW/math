% Options for packages loaded elsewhere
\PassOptionsToPackage{unicode}{hyperref}
\PassOptionsToPackage{hyphens}{url}
\PassOptionsToPackage{dvipsnames,svgnames,x11names}{xcolor}
%
\documentclass[
]{book}
\usepackage{amsmath,amssymb}
\usepackage{iftex}
\ifPDFTeX
  \usepackage[T1]{fontenc}
  \usepackage[utf8]{inputenc}
  \usepackage{textcomp} % provide euro and other symbols
\else % if luatex or xetex
  \usepackage{unicode-math} % this also loads fontspec
  \defaultfontfeatures{Scale=MatchLowercase}
  \defaultfontfeatures[\rmfamily]{Ligatures=TeX,Scale=1}
\fi
\usepackage{lmodern}
\ifPDFTeX\else
  % xetex/luatex font selection
\fi
% Use upquote if available, for straight quotes in verbatim environments
\IfFileExists{upquote.sty}{\usepackage{upquote}}{}
\IfFileExists{microtype.sty}{% use microtype if available
  \usepackage[]{microtype}
  \UseMicrotypeSet[protrusion]{basicmath} % disable protrusion for tt fonts
}{}
\makeatletter
\@ifundefined{KOMAClassName}{% if non-KOMA class
  \IfFileExists{parskip.sty}{%
    \usepackage{parskip}
  }{% else
    \setlength{\parindent}{0pt}
    \setlength{\parskip}{6pt plus 2pt minus 1pt}}
}{% if KOMA class
  \KOMAoptions{parskip=half}}
\makeatother
\usepackage{xcolor}
\usepackage[b5paper,tmargin=1.5cm,bmargin=1.5cm,lmargin=1.0cm,rmargin=1.0cm]{geometry}
\usepackage{color}
\usepackage{fancyvrb}
\newcommand{\VerbBar}{|}
\newcommand{\VERB}{\Verb[commandchars=\\\{\}]}
\DefineVerbatimEnvironment{Highlighting}{Verbatim}{commandchars=\\\{\}}
% Add ',fontsize=\small' for more characters per line
\usepackage{framed}
\definecolor{shadecolor}{RGB}{248,248,248}
\newenvironment{Shaded}{\begin{snugshade}}{\end{snugshade}}
\newcommand{\AlertTok}[1]{\textcolor[rgb]{0.94,0.16,0.16}{#1}}
\newcommand{\AnnotationTok}[1]{\textcolor[rgb]{0.56,0.35,0.01}{\textbf{\textit{#1}}}}
\newcommand{\AttributeTok}[1]{\textcolor[rgb]{0.13,0.29,0.53}{#1}}
\newcommand{\BaseNTok}[1]{\textcolor[rgb]{0.00,0.00,0.81}{#1}}
\newcommand{\BuiltInTok}[1]{#1}
\newcommand{\CharTok}[1]{\textcolor[rgb]{0.31,0.60,0.02}{#1}}
\newcommand{\CommentTok}[1]{\textcolor[rgb]{0.56,0.35,0.01}{\textit{#1}}}
\newcommand{\CommentVarTok}[1]{\textcolor[rgb]{0.56,0.35,0.01}{\textbf{\textit{#1}}}}
\newcommand{\ConstantTok}[1]{\textcolor[rgb]{0.56,0.35,0.01}{#1}}
\newcommand{\ControlFlowTok}[1]{\textcolor[rgb]{0.13,0.29,0.53}{\textbf{#1}}}
\newcommand{\DataTypeTok}[1]{\textcolor[rgb]{0.13,0.29,0.53}{#1}}
\newcommand{\DecValTok}[1]{\textcolor[rgb]{0.00,0.00,0.81}{#1}}
\newcommand{\DocumentationTok}[1]{\textcolor[rgb]{0.56,0.35,0.01}{\textbf{\textit{#1}}}}
\newcommand{\ErrorTok}[1]{\textcolor[rgb]{0.64,0.00,0.00}{\textbf{#1}}}
\newcommand{\ExtensionTok}[1]{#1}
\newcommand{\FloatTok}[1]{\textcolor[rgb]{0.00,0.00,0.81}{#1}}
\newcommand{\FunctionTok}[1]{\textcolor[rgb]{0.13,0.29,0.53}{\textbf{#1}}}
\newcommand{\ImportTok}[1]{#1}
\newcommand{\InformationTok}[1]{\textcolor[rgb]{0.56,0.35,0.01}{\textbf{\textit{#1}}}}
\newcommand{\KeywordTok}[1]{\textcolor[rgb]{0.13,0.29,0.53}{\textbf{#1}}}
\newcommand{\NormalTok}[1]{#1}
\newcommand{\OperatorTok}[1]{\textcolor[rgb]{0.81,0.36,0.00}{\textbf{#1}}}
\newcommand{\OtherTok}[1]{\textcolor[rgb]{0.56,0.35,0.01}{#1}}
\newcommand{\PreprocessorTok}[1]{\textcolor[rgb]{0.56,0.35,0.01}{\textit{#1}}}
\newcommand{\RegionMarkerTok}[1]{#1}
\newcommand{\SpecialCharTok}[1]{\textcolor[rgb]{0.81,0.36,0.00}{\textbf{#1}}}
\newcommand{\SpecialStringTok}[1]{\textcolor[rgb]{0.31,0.60,0.02}{#1}}
\newcommand{\StringTok}[1]{\textcolor[rgb]{0.31,0.60,0.02}{#1}}
\newcommand{\VariableTok}[1]{\textcolor[rgb]{0.00,0.00,0.00}{#1}}
\newcommand{\VerbatimStringTok}[1]{\textcolor[rgb]{0.31,0.60,0.02}{#1}}
\newcommand{\WarningTok}[1]{\textcolor[rgb]{0.56,0.35,0.01}{\textbf{\textit{#1}}}}
\usepackage{longtable,booktabs,array}
\usepackage{calc} % for calculating minipage widths
% Correct order of tables after \paragraph or \subparagraph
\usepackage{etoolbox}
\makeatletter
\patchcmd\longtable{\par}{\if@noskipsec\mbox{}\fi\par}{}{}
\makeatother
% Allow footnotes in longtable head/foot
\IfFileExists{footnotehyper.sty}{\usepackage{footnotehyper}}{\usepackage{footnote}}
\makesavenoteenv{longtable}
\usepackage{graphicx}
\makeatletter
\def\maxwidth{\ifdim\Gin@nat@width>\linewidth\linewidth\else\Gin@nat@width\fi}
\def\maxheight{\ifdim\Gin@nat@height>\textheight\textheight\else\Gin@nat@height\fi}
\makeatother
% Scale images if necessary, so that they will not overflow the page
% margins by default, and it is still possible to overwrite the defaults
% using explicit options in \includegraphics[width, height, ...]{}
\setkeys{Gin}{width=\maxwidth,height=\maxheight,keepaspectratio}
% Set default figure placement to htbp
\makeatletter
\def\fps@figure{htbp}
\makeatother
\setlength{\emergencystretch}{3em} % prevent overfull lines
\providecommand{\tightlist}{%
  \setlength{\itemsep}{0pt}\setlength{\parskip}{0pt}}
\setcounter{secnumdepth}{5}
\newlength{\cslhangindent}
\setlength{\cslhangindent}{1.5em}
\newlength{\csllabelwidth}
\setlength{\csllabelwidth}{3em}
\newlength{\cslentryspacingunit} % times entry-spacing
\setlength{\cslentryspacingunit}{\parskip}
\newenvironment{CSLReferences}[2] % #1 hanging-ident, #2 entry spacing
 {% don't indent paragraphs
  \setlength{\parindent}{0pt}
  % turn on hanging indent if param 1 is 1
  \ifodd #1
  \let\oldpar\par
  \def\par{\hangindent=\cslhangindent\oldpar}
  \fi
  % set entry spacing
  \setlength{\parskip}{#2\cslentryspacingunit}
 }%
 {}
\usepackage{calc}
\newcommand{\CSLBlock}[1]{#1\hfill\break}
\newcommand{\CSLLeftMargin}[1]{\parbox[t]{\csllabelwidth}{#1}}
\newcommand{\CSLRightInline}[1]{\parbox[t]{\linewidth - \csllabelwidth}{#1}\break}
\newcommand{\CSLIndent}[1]{\hspace{\cslhangindent}#1}
\usepackage{booktabs}
% \usepackage{fontspec} %這個可能原本文檔就已經有了,放入時候check一下
% \usepackage{CJKutf8}
% \usepackage[UTF8]{inputenc}
\usepackage{CJK}
% \usepackage{xeCJK}

%英文字體調整(有時候交中文文件可能有規定對應的英文字體)
% \setmainfont{Times New Roman}
% \setmainfont{Noto Sans}

%中文字體main跟mono都需要哦,最後面的SC是簡體中文,也可以改成TC,不過SC的破字會比較少
% \setCJKmainfont{NotoSansTC-Regular.otf}
% \setCJKmonofont{NotoSansTC-Regular.otf}

% \usepackage{bm}
\usepackage{amsmath,amssymb}
\usepackage{hyperref}
\hypersetup{
    colorlinks=true,
    linkcolor=blue,
    filecolor=magenta,      
    urlcolor=cyan
}

% to wrap the text inside the margins of the PDF document when using code chunks in bookdown
\usepackage{fvextra}
\DefineVerbatimEnvironment{Highlighting}{Verbatim}{breaklines,commandchars=\\\{\}}

% \usepackage[backend=bibtex]{biblatex}
% \usepackage[backend=biber]{biblatex}
% \usepackage[]{biblatex}
% \DeclarePrintbibliographyDefaults{heading=bibintoc}

% \let\oldpb\printbibliography
% \renewcommand{\printbibliography}{\oldpb[heading=bibintoc]}

\usepackage{tikz}
\usepackage{tikz-3dplot}
\usepackage{pgfplots}
\pgfplotsset{compat=1.15}

\usepackage{mathrsfs}
\usetikzlibrary{arrows}
% \pagestyle{empty}
% \newcommand{\degre}{\ensuremath{^\circ}}

\usepackage[all]{xy}

% LaTeX Error: Too deeply nested
% https://stackoverflow.com/questions/57945414/too-deeply-nested-at-just-fourth-nesting-level-using-pandoc-with-markdown
\usepackage{enumitem}
\setlistdepth{20}
\renewlist{itemize}{itemize}{20}
\renewlist{enumerate}{enumerate}{20}
\setlist[itemize]{label=$\cdot$}
\setlist[itemize,1]{label=\textbullet}
\setlist[itemize,2]{label=--}
\setlist[itemize,3]{label=*}
\ifLuaTeX
  \usepackage{selnolig}  % disable illegal ligatures
\fi
\IfFileExists{bookmark.sty}{\usepackage{bookmark}}{\usepackage{hyperref}}
\IfFileExists{xurl.sty}{\usepackage{xurl}}{} % add URL line breaks if available
\urlstyle{same}
\hypersetup{
  pdftitle={math},
  pdfauthor={Joey Yu Hsu},
  colorlinks=true,
  linkcolor={Maroon},
  filecolor={Maroon},
  citecolor={Blue},
  urlcolor={Blue},
  pdfcreator={LaTeX via pandoc}}

\title{math}
\author{Joey Yu Hsu}
\date{2024-02-16}

\usepackage{amsthm}
\newtheorem{theorem}{Theorem}[chapter]
\newtheorem{lemma}{Lemma}[chapter]
\newtheorem{corollary}{Corollary}[chapter]
\newtheorem{proposition}{Proposition}[chapter]
\newtheorem{conjecture}{Conjecture}[chapter]
\theoremstyle{definition}
\newtheorem{definition}{Definition}[chapter]
\theoremstyle{definition}
\newtheorem{example}{Example}[chapter]
\theoremstyle{definition}
\newtheorem{exercise}{Exercise}[chapter]
\theoremstyle{definition}
\newtheorem{hypothesis}{Hypothesis}[chapter]
\theoremstyle{remark}
\newtheorem*{remark}{Remark}
\newtheorem*{solution}{Solution}
\begin{document}
\maketitle

{
\hypersetup{linkcolor=}
\setcounter{tocdepth}{1}
\tableofcontents
}
\hypertarget{index}{%
\chapter*{index}\label{index}}
\addcontentsline{toc}{chapter}{index}

math on bookdown started on 2024/01/28

\hypertarget{part-a-minimal-book-example}{%
\part{A Minimal Book Example}\label{part-a-minimal-book-example}}

\hypertarget{about}{%
\chapter{About}\label{about}}

This is a \emph{sample} book written in \textbf{Markdown}. You can use anything that Pandoc's Markdown supports; for example, a math equation \(a^2 + b^2 = c^2\).

\hypertarget{usage}{%
\section{Usage}\label{usage}}

Each \textbf{bookdown} chapter is an .Rmd file, and each .Rmd file can contain one (and only one) chapter. A chapter \emph{must} start with a first-level heading: \texttt{\#\ A\ good\ chapter}, and can contain one (and only one) first-level heading.

Use second-level and higher headings within chapters like: \texttt{\#\#\ A\ short\ section} or \texttt{\#\#\#\ An\ even\ shorter\ section}.

The \texttt{index.Rmd} file is required, and is also your first book chapter. It will be the homepage when you render the book.

\hypertarget{render-book}{%
\section{Render book}\label{render-book}}

You can render the HTML version of this example book without changing anything:

\begin{enumerate}
\def\labelenumi{\arabic{enumi}.}
\item
  Find the \textbf{Build} pane in the RStudio IDE, and
\item
  Click on \textbf{Build Book}, then select your output format, or select ``All formats'' if you'd like to use multiple formats from the same book source files.
\end{enumerate}

Or build the book from the R console:

\begin{Shaded}
\begin{Highlighting}[]
\NormalTok{bookdown}\SpecialCharTok{::}\FunctionTok{render\_book}\NormalTok{()}
\end{Highlighting}
\end{Shaded}

To render this example to PDF as a \texttt{bookdown::pdf\_book}, you'll need to install XeLaTeX. You are recommended to install TinyTeX (which includes XeLaTeX): \url{https://yihui.org/tinytex/}.

\hypertarget{preview-book}{%
\section{Preview book}\label{preview-book}}

As you work, you may start a local server to live preview this HTML book. This preview will update as you edit the book when you save individual .Rmd files. You can start the server in a work session by using the RStudio add-in ``Preview book'', or from the R console:

\begin{Shaded}
\begin{Highlighting}[]
\NormalTok{bookdown}\SpecialCharTok{::}\FunctionTok{serve\_book}\NormalTok{()}
\end{Highlighting}
\end{Shaded}

\hypertarget{hello-bookdown}{%
\chapter{Hello bookdown}\label{hello-bookdown}}

All chapters start with a first-level heading followed by your chapter title, like the line above. There should be only one first-level heading (\texttt{\#}) per .Rmd file.

\hypertarget{a-section}{%
\section{A section}\label{a-section}}

All chapter sections start with a second-level (\texttt{\#\#}) or higher heading followed by your section title, like the sections above and below here. You can have as many as you want within a chapter.

\hypertarget{an-unnumbered-section}{%
\subsection*{An unnumbered section}\label{an-unnumbered-section}}
\addcontentsline{toc}{subsection}{An unnumbered section}

Chapters and sections are numbered by default. To un-number a heading, add a \texttt{\{.unnumbered\}} or the shorter \texttt{\{-\}} at the end of the heading, like in this section.

\hypertarget{cross}{%
\chapter{Cross-references}\label{cross}}

Cross-references make it easier for your readers to find and link to elements in your book.

\hypertarget{chapters-and-sub-chapters}{%
\section{Chapters and sub-chapters}\label{chapters-and-sub-chapters}}

There are two steps to cross-reference any heading:

\begin{enumerate}
\def\labelenumi{\arabic{enumi}.}
\tightlist
\item
  Label the heading: \texttt{\#\ Hello\ world\ \{\#nice-label\}}.

  \begin{itemize}
  \tightlist
  \item
    Leave the label off if you like the automated heading generated based on your heading title: for example, \texttt{\#\ Hello\ world} = \texttt{\#\ Hello\ world\ \{\#hello-world\}}.
  \item
    To label an un-numbered heading, use: \texttt{\#\ Hello\ world\ \{-\#nice-label\}} or \texttt{\{\#\ Hello\ world\ .unnumbered\}}.
  \end{itemize}
\item
  Next, reference the labeled heading anywhere in the text using \texttt{\textbackslash{}@ref(nice-label)}; for example, please see Chapter \ref{cross}.

  \begin{itemize}
  \tightlist
  \item
    If you prefer text as the link instead of a numbered reference use: \protect\hyperlink{cross}{any text you want can go here}.
  \end{itemize}
\end{enumerate}

\hypertarget{captioned-figures-and-tables}{%
\section{Captioned figures and tables}\label{captioned-figures-and-tables}}

Figures and tables \emph{with captions} can also be cross-referenced from elsewhere in your book using \texttt{\textbackslash{}@ref(fig:chunk-label)} and \texttt{\textbackslash{}@ref(tab:chunk-label)}, respectively.

See Figure \ref{fig:nice-fig}.

\begin{Shaded}
\begin{Highlighting}[]
\FunctionTok{par}\NormalTok{(}\AttributeTok{mar =} \FunctionTok{c}\NormalTok{(}\DecValTok{4}\NormalTok{, }\DecValTok{4}\NormalTok{, .}\DecValTok{1}\NormalTok{, .}\DecValTok{1}\NormalTok{))}
\FunctionTok{plot}\NormalTok{(pressure, }\AttributeTok{type =} \StringTok{\textquotesingle{}b\textquotesingle{}}\NormalTok{, }\AttributeTok{pch =} \DecValTok{19}\NormalTok{)}
\end{Highlighting}
\end{Shaded}

\begin{figure}

{\centering \includegraphics[width=0.8\linewidth]{02-cross-refs_files/figure-latex/nice-fig-1} 

}

\caption{Here is a nice figure!}\label{fig:nice-fig}
\end{figure}

Don't miss Table \ref{tab:nice-tab}.

\begin{Shaded}
\begin{Highlighting}[]
\NormalTok{knitr}\SpecialCharTok{::}\FunctionTok{kable}\NormalTok{(}
  \FunctionTok{head}\NormalTok{(pressure, }\DecValTok{10}\NormalTok{), }\AttributeTok{caption =} \StringTok{\textquotesingle{}Here is a nice table!\textquotesingle{}}\NormalTok{,}
  \AttributeTok{booktabs =} \ConstantTok{TRUE}
\NormalTok{)}
\end{Highlighting}
\end{Shaded}

\begin{table}

\caption{\label{tab:nice-tab}Here is a nice table!}
\centering
\begin{tabular}[t]{rr}
\toprule
temperature & pressure\\
\midrule
0 & 0.0002\\
20 & 0.0012\\
40 & 0.0060\\
60 & 0.0300\\
80 & 0.0900\\
\addlinespace
100 & 0.2700\\
120 & 0.7500\\
140 & 1.8500\\
160 & 4.2000\\
180 & 8.8000\\
\bottomrule
\end{tabular}
\end{table}

\hypertarget{parts}{%
\chapter{Parts}\label{parts}}

You can add parts to organize one or more book chapters together. Parts can be inserted at the top of an .Rmd file, before the first-level chapter heading in that same file.

Add a numbered part: \texttt{\#\ (PART)\ Act\ one\ \{-\}} (followed by \texttt{\#\ A\ chapter})

Add an unnumbered part: \texttt{\#\ (PART\textbackslash{}*)\ Act\ one\ \{-\}} (followed by \texttt{\#\ A\ chapter})

Add an appendix as a special kind of un-numbered part: \texttt{\#\ (APPENDIX)\ Other\ stuff\ \{-\}} (followed by \texttt{\#\ A\ chapter}). Chapters in an appendix are prepended with letters instead of numbers.

\hypertarget{footnotes-and-citations}{%
\chapter{Footnotes and citations}\label{footnotes-and-citations}}

\hypertarget{footnotes}{%
\section{Footnotes}\label{footnotes}}

Footnotes are put inside the square brackets after a caret \texttt{\^{}{[}{]}}. Like this one \footnote{This is a footnote.}.

\hypertarget{citations}{%
\section{Citations}\label{citations}}

Reference items in your bibliography file(s) using \texttt{@key}.

For example, we are using the \textbf{bookdown} package\textsuperscript{\protect\hyperlink{ref-R-bookdown}{1}} (check out the last code chunk in index.Rmd to see how this citation key was added) in this sample book, which was built on top of R Markdown and \textbf{knitr}\textsuperscript{\protect\hyperlink{ref-xie2015}{2}} (this citation was added manually in an external file book.bib).
Note that the \texttt{.bib} files need to be listed in the index.Rmd with the YAML \texttt{bibliography} key.

The RStudio Visual Markdown Editor can also make it easier to insert citations: \url{https://rstudio.github.io/visual-markdown-editing/\#/citations}

\hypertarget{blocks}{%
\chapter{Blocks}\label{blocks}}

\hypertarget{equations}{%
\section{Equations}\label{equations}}

Here is an equation.

\begin{equation} 
  f\left(k\right) = \binom{n}{k} p^k\left(1-p\right)^{n-k}
  \label{eq:binom}
\end{equation}

You may refer to using \texttt{\textbackslash{}@ref(eq:binom)}, like see Equation \eqref{eq:binom}.

\hypertarget{theorems-and-proofs}{%
\section{Theorems and proofs}\label{theorems-and-proofs}}

Labeled theorems can be referenced in text using \texttt{\textbackslash{}@ref(thm:tri)}, for example, check out this smart theorem \ref{thm:tri}.

\begin{theorem}
\protect\hypertarget{thm:tri}{}\label{thm:tri}For a right triangle, if \(c\) denotes the \emph{length} of the hypotenuse
and \(a\) and \(b\) denote the lengths of the \textbf{other} two sides, we have
\[a^2 + b^2 = c^2\]
\end{theorem}

Read more here \url{https://bookdown.org/yihui/bookdown/markdown-extensions-by-bookdown.html}.

\hypertarget{callout-blocks}{%
\section{Callout blocks}\label{callout-blocks}}

The R Markdown Cookbook provides more help on how to use custom blocks to design your own callouts: \url{https://bookdown.org/yihui/rmarkdown-cookbook/custom-blocks.html}

\hypertarget{sharing-your-book}{%
\chapter{Sharing your book}\label{sharing-your-book}}

\hypertarget{publishing}{%
\section{Publishing}\label{publishing}}

HTML books can be published online, see: \url{https://bookdown.org/yihui/bookdown/publishing.html}

\hypertarget{pages}{%
\section{404 pages}\label{pages}}

By default, users will be directed to a 404 page if they try to access a webpage that cannot be found. If you'd like to customize your 404 page instead of using the default, you may add either a \texttt{\_404.Rmd} or \texttt{\_404.md} file to your project root and use code and/or Markdown syntax.

\hypertarget{metadata-for-sharing}{%
\section{Metadata for sharing}\label{metadata-for-sharing}}

Bookdown HTML books will provide HTML metadata for social sharing on platforms like Twitter, Facebook, and LinkedIn, using information you provide in the \texttt{index.Rmd} YAML. To setup, set the \texttt{url} for your book and the path to your \texttt{cover-image} file. Your book's \texttt{title} and \texttt{description} are also used.

This \texttt{gitbook} uses the same social sharing data across all chapters in your book- all links shared will look the same.

Specify your book's source repository on GitHub using the \texttt{edit} key under the configuration options in the \texttt{\_output.yml} file, which allows users to suggest an edit by linking to a chapter's source file.

Read more about the features of this output format here:

\url{https://pkgs.rstudio.com/bookdown/reference/gitbook.html}

Or use:

\begin{Shaded}
\begin{Highlighting}[]
\NormalTok{?bookdown}\SpecialCharTok{::}\NormalTok{gitbook}
\end{Highlighting}
\end{Shaded}

\hypertarget{part-by-discipline}{%
\part{by discipline}\label{part-by-discipline}}

\hypertarget{test-cross-link}{%
\chapter{test cross-link}\label{test-cross-link}}

script\textsuperscript{superscript}\textsubscript{subscript}

\hypertarget{link-and-reference}{%
\section{link and reference}\label{link-and-reference}}

\begin{equation}
  E=mc^2
  \label{eq:emc}
\end{equation}

\texttt{\textbackslash{}@ref(nice-label)} \ref{nice-label}

\texttt{{[}link\ to\ partition{]}{[}partition{]}} \protect\hyperlink{partition}{link to partition}

\texttt{{[}partition{]}} \texttt{\textbackslash{}@ref(partition)}

\protect\hyperlink{partition}{partition} {[}\#partition{]} (\ref{partition}) @ref(\#partition)

\texttt{{[}equivalence\ class{]}} \texttt{\textbackslash{}@ref(equivalence\ class)}

\protect\hyperlink{equivalence-class}{equivalence class} {[}\#equivalence class{]} (@ref(equivalence class)) @ref(\#equivalence class)

{[}equivalence-class{]} {[}\#equivalence-class{]} (\ref{equivalence-class}) @ref(\#equivalence-class)

{[}equivalence-class.html{]} {[}equivalence-class.html\#equivalence-class{]} (@ref(equivalence-class.html)) @ref(equivalence-class.html\#equivalence-class)

\protect\hyperlink{equivalence-relation}{equivalence relation} {[}\#equivalence relation{]} (@ref(equivalence relation)) @ref(\#equivalence relation)

{[}equivalence-relation{]} {[}\#equivalence-relation{]} (\ref{equivalence-relation}) @ref(\#equivalence-relation)

{[}equivalence-relation.html{]} {[}equivalence-relation.html\#equivalence-relation{]} (@ref(equivalence-relation.html)) @ref(equivalence-relation.html\#equivalence-relation)

\hypertarget{number-and-reference-equations}{%
\section{number and reference equations}\label{number-and-reference-equations}}

\url{https://bookdown.org/yihui/rmarkdown/bookdown-markdown.html\#equations}

\texttt{\textbackslash{}\#eq:emc}
\texttt{\textbackslash{}@ref(eq:emc)}

\begin{align*}
\label{eq:eqclass}
 & C\text{ is an equivalence class of }a\text{ on }A\\
\Leftrightarrow & \left[a\right]_{\sim}=C=\left\{ x\middle|\begin{cases}
a\in A\\
x\in A\\
x\sim a\\
\sim\text{ is an equivalence relation over }A\times A=A^{2}
\end{cases}\right\} \subseteq A\ne\emptyset\\
\Leftrightarrow & \left[a\right]=\left[a\right]_{\sim}=\left\{ x\middle|\begin{cases}
a\in A\\
x\in A\\
x\sim a\\
\sim\text{ is an equivalence relation on }A
\end{cases}\right\} \subseteq A\ne\emptyset\\
\Rightarrow & \left[a\right]_{\sim}=\left\{ x\middle|x\sim a\right\} \subseteq A\ne\emptyset
\end{align*}

\url{https://bookdown.org/yihui/rmarkdown/bookdown-markdown.html\#cross-referencing}

This cross reference is the Fig. \ref{fig:parabola-arc-with-points}

\url{https://stackoverflow.com/questions/51595939/bookdown-cross-reference-figure-in-another-file}

I ran into the same issue and came up with this solution if you aim at compiling 2 different pdfs. It relies on LaTeX's xr package for cross references: \url{https://stackoverflow.com/a/52532269/576684}

\hypertarget{footnote}{%
\section{footnote}\label{footnote}}

noun\footnote{This is a footnote.}

\hypertarget{citation}{%
\section{citation}\label{citation}}

\url{https://stackoverflow.com/questions/48965247/use-csl-file-for-pdf-output-in-bookdown/49145699\#49145699}

citation 1\textsuperscript{\protect\hyperlink{ref-noauthor_bookdown_2019}{3}} citation 2\textsuperscript{\protect\hyperlink{ref-noauthor_bookdown_2019}{3}}

citation 3\textsuperscript{\protect\hyperlink{ref-ccjou2009}{4}} citation 4\textsuperscript{\protect\hyperlink{ref-ccjou2009}{4}}

\hypertarget{bookdown-environment-for-definition-theorem-proof}{%
\section{bookdown environment for definition, theorem, proof}\label{bookdown-environment-for-definition-theorem-proof}}

\url{https://bookdown.org/yihui/rmarkdown/bookdown-markdown.html}

\begin{theorem}[Theorem Name]
\protect\hypertarget{thm:label}{}\label{thm:label}Here is my theorem.
\end{theorem}

\begin{proof}[Proof Name]
Here is my proof.
\end{proof}

\begin{theorem}[Pythagorean theorem]
\protect\hypertarget{thm:pyth}{}\label{thm:pyth}For a right triangle, if \(c\) denotes the length of the hypotenuse
and \(a\) and \(b\) denote the lengths of the other two sides, we have

\[a^2 + \color{cyan}b^2 \overset{\ref{eq:emc}}= \color{red}{c^2} \]
\end{theorem}

\begin{definition}[Definition Name]
\protect\hypertarget{def:unnamed-chunk-2}{}\label{def:unnamed-chunk-2}Here is my definition.
\end{definition}

\protect\hyperlink{number-and-reference-equations}{number and reference equations}

\eqref{eq:eqclass}

\eqref{eq:emc}

\ref{thm:pyth}

\begin{figure}
\includegraphics[width=0.25\linewidth]{202401260000-test-cross-link_files/figure-latex/parabola-arc-with-points-1} \caption{parabola arc with points}\label{fig:parabola-arc-with-points}
\end{figure}

\hypertarget{test-cross-link-2}{%
\chapter{test cross-link 2}\label{test-cross-link-2}}

\hypertarget{nice-label}{%
\chapter{math}\label{nice-label}}

\protect\hyperlink{equivalence-relation}{equivalence relation} \ref{equivalence-relation}

\protect\hyperlink{equivalence-class}{equivalence class} \ref{equivalence-class}

\protect\hyperlink{partition}{partition} \ref{partition}

\hypertarget{equivalence-relation}{%
\chapter*{equivalence relation}\label{equivalence-relation}}
\addcontentsline{toc}{chapter}{equivalence relation}

\begin{CJK}{UTF8}{bsmi}等價關係 equivalence relation \label{def:equivalence-relation}
\end{CJK}
\begin{CJK}{UTF8}{bsmi}
\begin{align*}
 & R\text{ is an equivalence relation over }A\times B\\
\Leftrightarrow & \begin{cases}
R=\sim=\left\{ \left\langle x,y\right\rangle \middle|x\sim y\right\} \subseteq A\times B & \left(\text{e}\right)\text{equivalence 等價}\\
\vdots & \vdots
\end{cases}\\
\Leftrightarrow & \begin{cases}
R=\left\{ \left\langle x,y\right\rangle \middle|xRy\right\} \subseteq A\times B & \left(R\right)\text{relation}\\
\forall\left\langle x,y\right\rangle \in R\left(xRx\right) & \left(r\right)\text{reflexive}\\
\forall\left\langle x,y\right\rangle \in R\left(xRy\Rightarrow yRx\right) & \left(s\right)\text{symmetric}\\
\forall\left\langle x,y\right\rangle ,\left\langle y,z\right\rangle \in R\left(\begin{cases}
xRy\\
yRz
\end{cases}\Rightarrow xRz\right) & \left(t\right)\text{transitive}
\end{cases}\Leftrightarrow\begin{cases}
R=\left\{ \left\langle x,y\right\rangle \middle|xRy\right\} \subseteq A\times B & \text{關係}\\
\forall\left\langle x,y\right\rangle \in R\left(\left\langle x,x\right\rangle \in R\right) & \text{自反}\\
\forall\left\langle x,y\right\rangle \in R\left(\left\langle y,x\right\rangle \in R\right) & \text{對稱}\\
\forall\left\langle x,y\right\rangle ,\left\langle y,z\right\rangle \in R\left(\left\langle x,z\right\rangle \in R\right) & \text{遞移}
\end{cases}
\end{align*}
\end{CJK}

\hypertarget{equivalence-class}{%
\chapter*{equivalence class}\label{equivalence-class}}
\addcontentsline{toc}{chapter}{equivalence class}

\begin{align*}
 & C\text{ is an equivalence class of }a\text{ on }A\\
\Leftrightarrow & \left[a\right]_{\sim}=C=\left\{ x\middle|\begin{cases}
a\in A\\
x\in A\\
x\sim a\\
\sim\text{ is an equivalence relation over }A\times A=A^{2}
\end{cases}\right\} \subseteq A\ne\emptyset\\
\Leftrightarrow & \left[a\right]=\left[a\right]_{\sim}=\left\{ x\middle|\begin{cases}
a\in A\\
x\in A\\
x\sim a\\
\sim\text{ is an equivalence relation on }A
\end{cases}\right\} \subseteq A\ne\emptyset\\
\Rightarrow & \left[a\right]_{\sim}=\left\{ x\middle|x\sim a\right\} \subseteq A\ne\emptyset
\end{align*}

where the definition of \protect\hyperlink{equivalence-relation}{equivalence relation} can be found in \ref{equivalence-relation}.

\protect\hyperlink{number-and-reference-equations}{number and reference equations}

\eqref{eq:eqclass}

\eqref{eq:emc}

\ref{thm:pyth}

\hypertarget{partition}{%
\chapter*{partition}\label{partition}}
\addcontentsline{toc}{chapter}{partition}

\begin{align*}
 & \left\{ A_{i}\right\} _{i\in I}=\left\{ A_{i}\middle|i\in I\right\} \text{ is a partition of a set }A\\
\Leftrightarrow & \begin{cases}
\forall i\in I\left(A_{i}\ne\emptyset\right)\\
A=\bigcup\limits _{i\in I}A_{i}\\
\forall i,j\in I\left(i\ne j\Rightarrow A_{i}\cap A_{j}=\emptyset\right)
\end{cases}
\end{align*}

\url{https://proofwiki.org/wiki/Definition:Set_Partition}

\hypertarget{physics}{%
\chapter{physics}\label{physics}}

\hypertarget{plot}{%
\chapter{plot}\label{plot}}

\hypertarget{tikz}{%
\chapter*{TiKZ}\label{tikz}}
\addcontentsline{toc}{chapter}{TiKZ}

TiKZ and PGFplots

What's the relation between packages PGFplots and TikZ?

\url{https://tex.stackexchange.com/questions/285925/whats-the-relation-between-packages-pgfplots-and-tikz}

\url{https://www.youtube.com/watch?v=bQugbYq0BVA}

\url{https://www.youtube.com/watch?v=ft4Kg9emK1k\&list=PLg5nrpKdkk2DWcg3scb75AknF7DJXs8lk\&index=18}

\begin{Shaded}
\begin{Highlighting}[]
\KeywordTok{\textbackslash{}begin}\NormalTok{\{}\ExtensionTok{tikzpicture}\NormalTok{\}}
  \FunctionTok{\textbackslash{}def\textbackslash{}a}\NormalTok{\{1.5\} }\CommentTok{\% amplitude}
  \FunctionTok{\textbackslash{}def\textbackslash{}b}\NormalTok{\{2\}   }\CommentTok{\% frequency}
  \FunctionTok{\textbackslash{}draw}\NormalTok{[{-}\textgreater{}] ({-}0.2,0) {-}{-} (4.2,0) node[right, font=}\FunctionTok{\textbackslash{}small}\NormalTok{] \{}\SpecialStringTok{$x$}\NormalTok{\};}
  \FunctionTok{\textbackslash{}draw}\NormalTok{[{-}\textgreater{}] (0,{-}4) {-}{-} (0,0.5) node[above] \{}\SpecialStringTok{$y$}\NormalTok{\};}
  \FunctionTok{\textbackslash{}draw}\NormalTok{[domain=0:4,smooth,variable=}\FunctionTok{\textbackslash{}t}\NormalTok{,blue,thick] }
\NormalTok{    plot (\{}\FunctionTok{\textbackslash{}a}\NormalTok{ * (}\FunctionTok{\textbackslash{}b*\textbackslash{}t}\NormalTok{ {-} sin(deg(}\FunctionTok{\textbackslash{}b*\textbackslash{}t}\NormalTok{)))\},\{{-}}\FunctionTok{\textbackslash{}a}\NormalTok{ * (1 {-} cos(deg(}\FunctionTok{\textbackslash{}b*\textbackslash{}t}\NormalTok{)))\});}
  \CommentTok{\% \textbackslash{}node[above] at (2, 0.5) \{Brachistochrone Curve\};}
  \FunctionTok{\textbackslash{}node}\NormalTok{[above, font=}\FunctionTok{\textbackslash{}footnotesize}\NormalTok{] at (2, 1) \{Brachistochrone Curve\};}
  \FunctionTok{\textbackslash{}node}\NormalTok{[above, font=}\FunctionTok{\textbackslash{}footnotesize}\NormalTok{] at (2, 0) \{}\SpecialStringTok{$}\KeywordTok{\textbackslash{}begin}\NormalTok{\{}\ExtensionTok{aligned}\NormalTok{\}}
\SpecialStringTok{\& x=r(t{-}}\SpecialCharTok{\textbackslash{}sin}\SpecialStringTok{ t) }\SpecialCharTok{\textbackslash{}\textbackslash{}}
\SpecialStringTok{\& y=r(1{-}}\SpecialCharTok{\textbackslash{}cos}\SpecialStringTok{ t)}
\KeywordTok{\textbackslash{}end}\NormalTok{\{}\ExtensionTok{aligned}\NormalTok{\}}\SpecialStringTok{$}\NormalTok{\};}
\KeywordTok{\textbackslash{}end}\NormalTok{\{}\ExtensionTok{tikzpicture}\NormalTok{\}}
\end{Highlighting}
\end{Shaded}

\begin{figure}
\includegraphics[width=0.9\linewidth]{202401260003-plot_files/figure-latex/unnamed-chunk-4-1} \caption{Brachistochrone Curve}\label{fig:unnamed-chunk-4}
\end{figure}

\begin{figure}
\includegraphics[width=0.9\linewidth]{202401260003-plot_files/figure-latex/unnamed-chunk-5-1} \caption{Brachistochrone Curve}\label{fig:unnamed-chunk-5}
\end{figure}

\url{https://zhuanlan.zhihu.com/p/127155579?utm_psn=1741479950987960320}

1

\begin{Shaded}
\begin{Highlighting}[]
\KeywordTok{\textbackslash{}begin}\NormalTok{\{}\ExtensionTok{tikzpicture}\NormalTok{\}}
  \FunctionTok{\textbackslash{}draw}\NormalTok{ ({-}1,1){-}{-}(0,0){-}{-}(1,2);}
\KeywordTok{\textbackslash{}end}\NormalTok{\{}\ExtensionTok{tikzpicture}\NormalTok{\}}
\end{Highlighting}
\end{Shaded}

\begin{figure}
\includegraphics[width=0.5\linewidth]{202401260003-plot_files/figure-latex/unnamed-chunk-7-1} \end{figure}

\begin{figure}
\includegraphics[width=0.5\linewidth]{202401260003-plot_files/figure-latex/unnamed-chunk-8-1} \end{figure}

\begin{figure}
\includegraphics[width=1\linewidth]{202401260003-plot_files/figure-latex/unnamed-chunk-9-1} \end{figure}

2

\begin{figure}
\includegraphics[width=0.9\linewidth]{202401260003-plot_files/figure-latex/unnamed-chunk-10-1} \end{figure}

3

\begin{figure}
\includegraphics[width=0.25\linewidth]{202401260003-plot_files/figure-latex/unnamed-chunk-11-1} \end{figure}

\begin{Shaded}
\begin{Highlighting}[]
\KeywordTok{\textbackslash{}begin}\NormalTok{\{}\ExtensionTok{tikzpicture}\NormalTok{\}}
  \FunctionTok{\textbackslash{}draw}\NormalTok{[rounded corners] ({-}1,1){-}{-}(0,0){-}{-}(1,2){-}{-}({-}1,1);}
\KeywordTok{\textbackslash{}end}\NormalTok{\{}\ExtensionTok{tikzpicture}\NormalTok{\}}
\end{Highlighting}
\end{Shaded}

\begin{figure}
\includegraphics[width=0.25\linewidth]{202401260003-plot_files/figure-latex/unnamed-chunk-13-1} \caption{rounded corner pseudo-closed triangle}\label{fig:unnamed-chunk-13}
\end{figure}

\begin{Shaded}
\begin{Highlighting}[]
\KeywordTok{\textbackslash{}begin}\NormalTok{\{}\ExtensionTok{tikzpicture}\NormalTok{\}}
  \FunctionTok{\textbackslash{}draw}\NormalTok{[rounded corners] ({-}1,1){-}{-}(0,0){-}{-}(1,2){-}{-}cycle;}
\KeywordTok{\textbackslash{}end}\NormalTok{\{}\ExtensionTok{tikzpicture}\NormalTok{\}}
\end{Highlighting}
\end{Shaded}

\begin{figure}
\includegraphics[width=0.25\linewidth]{202401260003-plot_files/figure-latex/unnamed-chunk-15-1} \caption{rounded corner triangle}\label{fig:unnamed-chunk-15}
\end{figure}

\begin{figure}
\includegraphics[width=0.25\linewidth]{202401260003-plot_files/figure-latex/unnamed-chunk-16-1} \caption{triangle vs. pseudo-closed triangle}\label{fig:unnamed-chunk-16}
\end{figure}

\begin{Shaded}
\begin{Highlighting}[]
\KeywordTok{\textbackslash{}begin}\NormalTok{\{}\ExtensionTok{tikzpicture}\NormalTok{\}}
  \FunctionTok{\textbackslash{}draw}\NormalTok{ (0,0) rectangle (4,2);}
\KeywordTok{\textbackslash{}end}\NormalTok{\{}\ExtensionTok{tikzpicture}\NormalTok{\}}
\end{Highlighting}
\end{Shaded}

\begin{figure}
\includegraphics[width=0.25\linewidth]{202401260003-plot_files/figure-latex/unnamed-chunk-18-1} \caption{rectangle}\label{fig:unnamed-chunk-18}
\end{figure}

\begin{Shaded}
\begin{Highlighting}[]
\KeywordTok{\textbackslash{}begin}\NormalTok{\{}\ExtensionTok{tikzpicture}\NormalTok{\}}
  \FunctionTok{\textbackslash{}draw}\NormalTok{ (0,0) rectangle (2,2);}
\KeywordTok{\textbackslash{}end}\NormalTok{\{}\ExtensionTok{tikzpicture}\NormalTok{\}}
\end{Highlighting}
\end{Shaded}

\begin{figure}
\includegraphics[width=0.25\linewidth]{202401260003-plot_files/figure-latex/unnamed-chunk-20-1} \caption{square}\label{fig:unnamed-chunk-20}
\end{figure}

\begin{Shaded}
\begin{Highlighting}[]
\KeywordTok{\textbackslash{}begin}\NormalTok{\{}\ExtensionTok{tikzpicture}\NormalTok{\}}
  \FunctionTok{\textbackslash{}draw}\NormalTok{ (0,0) circle (1);}
\KeywordTok{\textbackslash{}end}\NormalTok{\{}\ExtensionTok{tikzpicture}\NormalTok{\}}
\end{Highlighting}
\end{Shaded}

\begin{figure}
\includegraphics[width=0.25\linewidth]{202401260003-plot_files/figure-latex/unnamed-chunk-22-1} \caption{circle}\label{fig:unnamed-chunk-22}
\end{figure}

\begin{Shaded}
\begin{Highlighting}[]
\KeywordTok{\textbackslash{}begin}\NormalTok{\{}\ExtensionTok{tikzpicture}\NormalTok{\}}
  \FunctionTok{\textbackslash{}draw}\NormalTok{ (0,0) circle (1);}
  \FunctionTok{\textbackslash{}draw}\NormalTok{ (0,0) rectangle (2,2);}
\KeywordTok{\textbackslash{}end}\NormalTok{\{}\ExtensionTok{tikzpicture}\NormalTok{\}}
\end{Highlighting}
\end{Shaded}

\begin{figure}
\includegraphics[width=0.25\linewidth]{202401260003-plot_files/figure-latex/unnamed-chunk-24-1} \caption{circle and square}\label{fig:unnamed-chunk-24}
\end{figure}

\begin{Shaded}
\begin{Highlighting}[]
\KeywordTok{\textbackslash{}begin}\NormalTok{\{}\ExtensionTok{tikzpicture}\NormalTok{\}}
  \FunctionTok{\textbackslash{}draw}\NormalTok{ (1,1) ellipse (2 and 1);}
\KeywordTok{\textbackslash{}end}\NormalTok{\{}\ExtensionTok{tikzpicture}\NormalTok{\}}
\end{Highlighting}
\end{Shaded}

\begin{figure}
\includegraphics[width=0.25\linewidth]{202401260003-plot_files/figure-latex/unnamed-chunk-26-1} \caption{ellipse}\label{fig:unnamed-chunk-26}
\end{figure}

\begin{Shaded}
\begin{Highlighting}[]
\KeywordTok{\textbackslash{}begin}\NormalTok{\{}\ExtensionTok{tikzpicture}\NormalTok{\}}
  \FunctionTok{\textbackslash{}draw}\NormalTok{ (1 ,1) arc (0:270:1);}
  \FunctionTok{\textbackslash{}draw}\NormalTok{ (6 ,1) arc (0:270:2 and 1);}
\KeywordTok{\textbackslash{}end}\NormalTok{\{}\ExtensionTok{tikzpicture}\NormalTok{\}}
\end{Highlighting}
\end{Shaded}

\begin{figure}
\includegraphics[width=0.25\linewidth]{202401260003-plot_files/figure-latex/unnamed-chunk-28-1} \caption{circle and ellipse arcs}\label{fig:unnamed-chunk-28}
\end{figure}

\begin{Shaded}
\begin{Highlighting}[]
\KeywordTok{\textbackslash{}begin}\NormalTok{\{}\ExtensionTok{tikzpicture}\NormalTok{\}}
  \FunctionTok{\textbackslash{}draw}\NormalTok{ ({-}1,1) parabola bend (0,0) (2,4);}
\KeywordTok{\textbackslash{}end}\NormalTok{\{}\ExtensionTok{tikzpicture}\NormalTok{\}}
\end{Highlighting}
\end{Shaded}

\begin{figure}
\includegraphics[width=0.25\linewidth]{202401260003-plot_files/figure-latex/unnamed-chunk-30-1} \caption{parabola arc}\label{fig:unnamed-chunk-30}
\end{figure}

\begin{Shaded}
\begin{Highlighting}[]
\KeywordTok{\textbackslash{}begin}\NormalTok{\{}\ExtensionTok{tikzpicture}\NormalTok{\}}
  \FunctionTok{\textbackslash{}draw}\NormalTok{ ({-}1,1) parabola bend (0,0) (2,4);}
  \FunctionTok{\textbackslash{}filldraw}
\NormalTok{    ({-}1,1) circle (.05)}
\NormalTok{    ( 0,0) circle (.05)}
\NormalTok{    ( 1,1) circle (.05)}
\NormalTok{    ( 2,4) circle (.05);}
\KeywordTok{\textbackslash{}end}\NormalTok{\{}\ExtensionTok{tikzpicture}\NormalTok{\}}
\end{Highlighting}
\end{Shaded}

\begin{figure}
\includegraphics[width=0.25\linewidth]{202401260003-plot_files/figure-latex/unnamed-chunk-32-1} \caption{parabola arc with points}\label{fig:unnamed-chunk-32}
\end{figure}

\begin{Shaded}
\begin{Highlighting}[]
\KeywordTok{\textbackslash{}begin}\NormalTok{\{}\ExtensionTok{tikzpicture}\NormalTok{\}}
  \FunctionTok{\textbackslash{}draw}\NormalTok{ [step=20pt] (0,0) grid (3,2);}
  \FunctionTok{\textbackslash{}draw}\NormalTok{ [help lines ,step=20pt] (4,0) grid (7,2);}
\KeywordTok{\textbackslash{}end}\NormalTok{\{}\ExtensionTok{tikzpicture}\NormalTok{\}}
\end{Highlighting}
\end{Shaded}

\begin{figure}
\includegraphics[width=0.75\linewidth]{202401260003-plot_files/figure-latex/unnamed-chunk-34-1} \caption{grid and help lines}\label{fig:unnamed-chunk-34}
\end{figure}

\begin{figure}
\includegraphics[width=0.75\linewidth]{202401260003-plot_files/figure-latex/unnamed-chunk-35-1} \caption{grid and help lines}\label{fig:unnamed-chunk-35}
\end{figure}

\begin{Shaded}
\begin{Highlighting}[]
\KeywordTok{\textbackslash{}begin}\NormalTok{\{}\ExtensionTok{tikzpicture}\NormalTok{\}[scale=0.25]}
  \FunctionTok{\textbackslash{}draw}\NormalTok{ [{-}\textgreater{}] (0,0){-}{-}(9,0);}
  \FunctionTok{\textbackslash{}draw}\NormalTok{ [\textless{}{-}] (0,1){-}{-}(9,1);}
  \FunctionTok{\textbackslash{}draw}\NormalTok{ [\textless{}{-}\textgreater{}] (0,2){-}{-}(9,2);}
  \FunctionTok{\textbackslash{}draw}\NormalTok{ [\textgreater{}{-}\textgreater{}\textgreater{}] (0,3){-}{-}(9,3);}
  \FunctionTok{\textbackslash{}draw}\NormalTok{ [|\textless{}{-}\textgreater{}|] (0,4){-}{-}(9,4);}
\KeywordTok{\textbackslash{}end}\NormalTok{\{}\ExtensionTok{tikzpicture}\NormalTok{\}}
\end{Highlighting}
\end{Shaded}

\begin{figure}
\includegraphics[width=0.75\linewidth]{202401260003-plot_files/figure-latex/unnamed-chunk-37-1} \caption{arrows}\label{fig:unnamed-chunk-37}
\end{figure}

\begin{Shaded}
\begin{Highlighting}[]
\KeywordTok{\textbackslash{}begin}\NormalTok{\{}\ExtensionTok{tikzpicture}\NormalTok{\}}
  \FunctionTok{\textbackslash{}draw}\NormalTok{ [line width =2pt] (0,6){-}{-}(9,6); }
  \FunctionTok{\textbackslash{}draw}\NormalTok{ [dotted]          (0,5){-}{-}(9,5); }
  \FunctionTok{\textbackslash{}draw}\NormalTok{ [densely dotted]  (0,4){-}{-}(9,4); }
  \FunctionTok{\textbackslash{}draw}\NormalTok{ [loosely dotted]  (0,3){-}{-}(9,3); }
  \FunctionTok{\textbackslash{}draw}\NormalTok{ [dashed]          (0,2){-}{-}(9,2); }
  \FunctionTok{\textbackslash{}draw}\NormalTok{ [densely dashed]  (0,1){-}{-}(9,1); }
  \FunctionTok{\textbackslash{}draw}\NormalTok{ [loosely dashed]  (0,0){-}{-}(9,0);}
\KeywordTok{\textbackslash{}end}\NormalTok{\{}\ExtensionTok{tikzpicture}\NormalTok{\}}
\end{Highlighting}
\end{Shaded}

\begin{figure}
\includegraphics[width=0.75\linewidth]{202401260003-plot_files/figure-latex/unnamed-chunk-39-1} \caption{arrows}\label{fig:unnamed-chunk-39}
\end{figure}

\begin{Shaded}
\begin{Highlighting}[]
\KeywordTok{\textbackslash{}begin}\NormalTok{\{}\ExtensionTok{tikzpicture}\NormalTok{\}[dline/.style=\{color= blue, line width=2pt\}]}
  \FunctionTok{\textbackslash{}draw}\NormalTok{[dline] (0,0){-}{-}(9,0); }
\KeywordTok{\textbackslash{}end}\NormalTok{\{}\ExtensionTok{tikzpicture}\NormalTok{\}}
\end{Highlighting}
\end{Shaded}

\begin{figure}
\includegraphics[width=0.75\linewidth]{202401260003-plot_files/figure-latex/unnamed-chunk-41-1} \caption{head styling}\label{fig:unnamed-chunk-41}
\end{figure}

\begin{Shaded}
\begin{Highlighting}[]
\KeywordTok{\textbackslash{}begin}\NormalTok{\{}\ExtensionTok{tikzpicture}\NormalTok{\}}
  \FunctionTok{\textbackslash{}draw}\NormalTok{ (0,0) rectangle (2,2);}
  \FunctionTok{\textbackslash{}draw}\NormalTok{[shift=\{( 3, 0)\}] (0,0) rectangle (2,2);}
  \FunctionTok{\textbackslash{}draw}\NormalTok{[shift=\{( 0, 3)\}] (0,0) rectangle (2,2);}
  \FunctionTok{\textbackslash{}draw}\NormalTok{[shift=\{( 0,{-}3)\}] (0,0) rectangle (2,2);}
  \FunctionTok{\textbackslash{}draw}\NormalTok{[shift=\{({-}3, 0)\}] (0,0) rectangle (2,2);}
  \FunctionTok{\textbackslash{}draw}\NormalTok{[shift=\{( 3, 3)\}] (0,0) rectangle (2,2);}
  \FunctionTok{\textbackslash{}draw}\NormalTok{[shift=\{({-}3, 3)\}] (0,0) rectangle (2,2);}
  \FunctionTok{\textbackslash{}draw}\NormalTok{[shift=\{( 3,{-}3)\}] (0,0) rectangle (2,2);}
  \FunctionTok{\textbackslash{}draw}\NormalTok{[shift=\{({-}3,{-}3)\}] (0,0) rectangle (2,2);}
\KeywordTok{\textbackslash{}end}\NormalTok{\{}\ExtensionTok{tikzpicture}\NormalTok{\}}
\end{Highlighting}
\end{Shaded}

\begin{figure}
\includegraphics[width=0.75\linewidth]{202401260003-plot_files/figure-latex/unnamed-chunk-43-1} \caption{transform: shift}\label{fig:unnamed-chunk-43}
\end{figure}

\begin{Shaded}
\begin{Highlighting}[]
\KeywordTok{\textbackslash{}begin}\NormalTok{\{}\ExtensionTok{tikzpicture}\NormalTok{\}}
  \FunctionTok{\textbackslash{}draw}\NormalTok{ (0,0) rectangle (2,2);}
  \FunctionTok{\textbackslash{}draw}\NormalTok{[xshift= 100pt] (0,0) rectangle (2,2);}
  \FunctionTok{\textbackslash{}draw}\NormalTok{[xshift={-}100pt] (0,0) rectangle (2,2);}
  \FunctionTok{\textbackslash{}draw}\NormalTok{[yshift= 100pt] (0,0) rectangle (2,2);}
  \FunctionTok{\textbackslash{}draw}\NormalTok{[yshift={-}100pt] (0,0) rectangle (2,2);}
\KeywordTok{\textbackslash{}end}\NormalTok{\{}\ExtensionTok{tikzpicture}\NormalTok{\}}
\end{Highlighting}
\end{Shaded}

\begin{figure}
\includegraphics[width=0.75\linewidth]{202401260003-plot_files/figure-latex/unnamed-chunk-45-1} \caption{transform: shift x, y}\label{fig:unnamed-chunk-45}
\end{figure}

\begin{Shaded}
\begin{Highlighting}[]
\KeywordTok{\textbackslash{}begin}\NormalTok{\{}\ExtensionTok{tikzpicture}\NormalTok{\}}
  \FunctionTok{\textbackslash{}draw}\NormalTok{ (0,0) rectangle (2,2);}
  \FunctionTok{\textbackslash{}draw}\NormalTok{[xshift= 100pt, xscale=1.5] (0,0) rectangle (2,2);}
  \FunctionTok{\textbackslash{}draw}\NormalTok{[yshift= 100pt, xscale=0.5] (0,0) rectangle (2,2);}
  \FunctionTok{\textbackslash{}draw}\NormalTok{[xshift={-}100pt, yscale=1.5] (0,0) rectangle (2,2);}
  \FunctionTok{\textbackslash{}draw}\NormalTok{[yshift={-}100pt, yscale=0.5] (0,0) rectangle (2,2);}
\KeywordTok{\textbackslash{}end}\NormalTok{\{}\ExtensionTok{tikzpicture}\NormalTok{\}}
\end{Highlighting}
\end{Shaded}

\begin{figure}
\includegraphics[width=0.75\linewidth]{202401260003-plot_files/figure-latex/unnamed-chunk-47-1} \caption{transform: scale x, y}\label{fig:unnamed-chunk-47}
\end{figure}

\begin{Shaded}
\begin{Highlighting}[]
\KeywordTok{\textbackslash{}begin}\NormalTok{\{}\ExtensionTok{tikzpicture}\NormalTok{\}}
  \FunctionTok{\textbackslash{}draw}\NormalTok{ (0,0) rectangle (2,2);}
  \FunctionTok{\textbackslash{}draw}\NormalTok{[xshift= 100pt, xscale=1.5] (0,0) rectangle (2,2);}
  \FunctionTok{\textbackslash{}draw}\NormalTok{[yshift= 100pt, yscale=1.5] (0,0) rectangle (2,2);}
  \FunctionTok{\textbackslash{}draw}\NormalTok{[xshift={-}100pt, xscale=0.5] (0,0) rectangle (2,2);}
  \FunctionTok{\textbackslash{}draw}\NormalTok{[yshift={-}100pt, yscale=0.5] (0,0) rectangle (2,2);}
\KeywordTok{\textbackslash{}end}\NormalTok{\{}\ExtensionTok{tikzpicture}\NormalTok{\}}
\end{Highlighting}
\end{Shaded}

\begin{figure}
\includegraphics[width=0.75\linewidth]{202401260003-plot_files/figure-latex/unnamed-chunk-49-1} \caption{transform: scale}\label{fig:unnamed-chunk-49}
\end{figure}

\begin{Shaded}
\begin{Highlighting}[]
\KeywordTok{\textbackslash{}begin}\NormalTok{\{}\ExtensionTok{tikzpicture}\NormalTok{\}}
  \FunctionTok{\textbackslash{}draw}\NormalTok{ (0,0) rectangle (2,2);}
  \FunctionTok{\textbackslash{}draw}\NormalTok{[xshift=125pt,rotate=45] (0,0) rectangle (2,2);}
  \FunctionTok{\textbackslash{}draw}\NormalTok{[xshift=175pt,rotate around=\{45:(2 ,2)\}] (0,0) rectangle (2,2);}
\KeywordTok{\textbackslash{}end}\NormalTok{\{}\ExtensionTok{tikzpicture}\NormalTok{\}}
\end{Highlighting}
\end{Shaded}

\begin{figure}
\includegraphics[width=0.75\linewidth]{202401260003-plot_files/figure-latex/unnamed-chunk-51-1} \caption{transform: rotate}\label{fig:unnamed-chunk-51}
\end{figure}

\begin{Shaded}
\begin{Highlighting}[]
\KeywordTok{\textbackslash{}begin}\NormalTok{\{}\ExtensionTok{tikzpicture}\NormalTok{\}}
  \FunctionTok{\textbackslash{}draw}\NormalTok{ (0,0) rectangle (2,2);}
  \FunctionTok{\textbackslash{}draw}\NormalTok{[xshift=70pt,xslant=1] (0,0) rectangle (2,2);}
  \FunctionTok{\textbackslash{}draw}\NormalTok{[yshift=70pt,yslant=1] (0,0) rectangle (2,2);}
\KeywordTok{\textbackslash{}end}\NormalTok{\{}\ExtensionTok{tikzpicture}\NormalTok{\}}
\end{Highlighting}
\end{Shaded}

\begin{figure}
\includegraphics[width=0.75\linewidth]{202401260003-plot_files/figure-latex/unnamed-chunk-53-1} \caption{transform: slant}\label{fig:unnamed-chunk-53}
\end{figure}

\begin{Shaded}
\begin{Highlighting}[]
\FunctionTok{\textbackslash{}tikzset}\NormalTok{\{}
\NormalTok{  box/.style=\{}
\NormalTok{    draw=blue,}
\NormalTok{    rectangle,}
\NormalTok{    rounded corners=5pt,}
\NormalTok{    minimum width=50pt,}
\NormalTok{    minimum height=20pt,}
\NormalTok{    inner sep=5pt}
\NormalTok{  \}}
\NormalTok{\}}
\KeywordTok{\textbackslash{}begin}\NormalTok{\{}\ExtensionTok{tikzpicture}\NormalTok{\}}
  \FunctionTok{\textbackslash{}node}\NormalTok{[box] (1) at(0,0) \{1\};}
  \FunctionTok{\textbackslash{}node}\NormalTok{[box] (2) at(4,0) \{2\};}
  \FunctionTok{\textbackslash{}node}\NormalTok{[box] (3) at(8,0) \{3\};}
  \FunctionTok{\textbackslash{}draw}\NormalTok{[{-}\textgreater{}] (1){-}{-}(2);}
  \FunctionTok{\textbackslash{}draw}\NormalTok{[{-}\textgreater{}] (2){-}{-}(3);}
  \FunctionTok{\textbackslash{}node}\NormalTok{ at(2,1) \{a\};}
  \FunctionTok{\textbackslash{}node}\NormalTok{ at(6,1) \{b\};}
\KeywordTok{\textbackslash{}end}\NormalTok{\{}\ExtensionTok{tikzpicture}\NormalTok{\}}
\end{Highlighting}
\end{Shaded}

\begin{figure}
\includegraphics[width=0.75\linewidth]{202401260003-plot_files/figure-latex/unnamed-chunk-55-1} \caption{flowchart}\label{fig:unnamed-chunk-55}
\end{figure}

\begin{Shaded}
\begin{Highlighting}[]
\FunctionTok{\textbackslash{}tikzset}\NormalTok{\{}
\NormalTok{  box/.style=\{}
\NormalTok{    draw=blue,}
\NormalTok{    fill=blue!20,}
\NormalTok{    rectangle,}
\NormalTok{    rounded corners=5pt,}
\NormalTok{    minimum height=20pt,}
\NormalTok{    inner sep=5pt}
\NormalTok{  \}}
\NormalTok{\}}
\KeywordTok{\textbackslash{}begin}\NormalTok{\{}\ExtensionTok{tikzpicture}\NormalTok{\}}
  \FunctionTok{\textbackslash{}node}\NormalTok{[box] \{1\}}
\NormalTok{      child \{node[box] \{2\}\}}
\NormalTok{      child \{node[box] \{3\}}
\NormalTok{          child \{node[box] \{4\}\}}
\NormalTok{          child \{node[box] \{5\}\}}
\NormalTok{          child \{node[box] \{6\}\}}
\NormalTok{      \};}
\KeywordTok{\textbackslash{}end}\NormalTok{\{}\ExtensionTok{tikzpicture}\NormalTok{\}}
\end{Highlighting}
\end{Shaded}

\begin{figure}
\includegraphics[width=0.75\linewidth]{202401260003-plot_files/figure-latex/unnamed-chunk-57-1} \caption{tree}\label{fig:unnamed-chunk-57}
\end{figure}

\begin{Shaded}
\begin{Highlighting}[]
\KeywordTok{\textbackslash{}begin}\NormalTok{\{}\ExtensionTok{tikzpicture}\NormalTok{\}}
  \FunctionTok{\textbackslash{}draw}\NormalTok{[{-}\textgreater{}] ({-}0.2,0){-}{-}(6,0) node[right] \{}\SpecialStringTok{$x$}\NormalTok{\};}
  \FunctionTok{\textbackslash{}draw}\NormalTok{[{-}\textgreater{}] (0,{-}0.2){-}{-}(0,6) node[above] \{}\SpecialStringTok{$f(x)$}\NormalTok{\};}
  \FunctionTok{\textbackslash{}draw}\NormalTok{[domain=0:4] plot (}\FunctionTok{\textbackslash{}x}\NormalTok{ ,\{0.1* exp(}\FunctionTok{\textbackslash{}x}\NormalTok{)\}) node[right] \{}\SpecialStringTok{$f(x)=}\SpecialCharTok{\textbackslash{}frac}\SpecialStringTok{\{1\}\{10\}e\^{}x$}\NormalTok{\};}
\KeywordTok{\textbackslash{}end}\NormalTok{\{}\ExtensionTok{tikzpicture}\NormalTok{\}}
\end{Highlighting}
\end{Shaded}

\begin{figure}
\includegraphics[width=0.75\linewidth]{202401260003-plot_files/figure-latex/unnamed-chunk-59-1} \caption{tree}\label{fig:unnamed-chunk-59}
\end{figure}

\url{https://stackoverflow.com/questions/64897575/tikz-libraries-in-bookdown}

It turns out that you can simply put the \texttt{\textbackslash{}usetikzlibrary\{...\}} command directly before the \texttt{\textbackslash{}begin\{tikzpicture\}} and everything works fine :)

\url{https://stackoverflow.com/questions/56211210/r-markdown-document-with-html-docx-output-using-latex-package-bbm}

\url{https://tex.stackexchange.com/questions/171711/how-to-include-latex-package-in-r-markdown}

\hypertarget{d}{%
\section{3D}\label{d}}

\url{https://zhuanlan.zhihu.com/p/431732330?utm_psn=1741857547550638080}

\begin{Shaded}
\begin{Highlighting}[]
\KeywordTok{\textbackslash{}begin}\NormalTok{\{}\ExtensionTok{tikzpicture}\NormalTok{\}}
  \FunctionTok{\textbackslash{}coordinate}\NormalTok{ (A) at ( 1, 1, 1);}
  \FunctionTok{\textbackslash{}coordinate}\NormalTok{ (B) at ( 1, 1,{-}1);}
  \FunctionTok{\textbackslash{}coordinate}\NormalTok{ (C) at ( 1,{-}1,{-}1);}
  \FunctionTok{\textbackslash{}coordinate}\NormalTok{ (D) at ( 1,{-}1, 1);}
  \FunctionTok{\textbackslash{}coordinate}\NormalTok{ (E) at ({-}1,{-}1, 1);}
  \FunctionTok{\textbackslash{}coordinate}\NormalTok{ (F) at ({-}1,{-}1,{-}1);}
  \FunctionTok{\textbackslash{}coordinate}\NormalTok{ (G) at ({-}1, 1,{-}1);}
  \FunctionTok{\textbackslash{}coordinate}\NormalTok{ (H) at ({-}1, 1, 1);}
  \FunctionTok{\textbackslash{}draw}\NormalTok{ (A) node[right=1pt] \{}\SpecialStringTok{$A$}\NormalTok{\}{-}{-}}
\NormalTok{        (B) node[right=1pt] \{}\SpecialStringTok{$B$}\NormalTok{\}{-}{-}}
\NormalTok{        (C) node[right=1pt] \{}\SpecialStringTok{$C$}\NormalTok{\}{-}{-}}
\NormalTok{        (D) node[right=1pt] \{}\SpecialStringTok{$D$}\NormalTok{\}{-}{-}}
\NormalTok{        (E) node[left= 1pt] \{}\SpecialStringTok{$E$}\NormalTok{\}{-}{-}}
\NormalTok{        (F) node[right=1pt] \{}\SpecialStringTok{$F$}\NormalTok{\}{-}{-}}
\NormalTok{        (G) node[right=1pt] \{}\SpecialStringTok{$G$}\NormalTok{\}{-}{-}}
\NormalTok{        (H) node[left= 1pt] \{}\SpecialStringTok{$H$}\NormalTok{\}{-}{-}}
\NormalTok{        (A) node[right=1pt] \{}\SpecialStringTok{$A$}\NormalTok{\};}
\KeywordTok{\textbackslash{}end}\NormalTok{\{}\ExtensionTok{tikzpicture}\NormalTok{\}}
\end{Highlighting}
\end{Shaded}

\begin{figure}
\includegraphics[width=0.75\linewidth]{202401260003-plot_files/figure-latex/unnamed-chunk-61-1} \caption{cube}\label{fig:unnamed-chunk-61}
\end{figure}

\url{https://tex.stackexchange.com/questions/388621/optimizing-perspective-tikz-graphic}

\begin{figure}
\includegraphics[width=0.75\linewidth]{202401260003-plot_files/figure-latex/unnamed-chunk-62-1} \caption{cube rotate}\label{fig:unnamed-chunk-62}
\end{figure}

\begin{figure}
\includegraphics[width=0.75\linewidth]{202401260003-plot_files/figure-latex/unnamed-chunk-63-1} \caption{cube rotate}\label{fig:unnamed-chunk-63}
\end{figure}

\url{https://github.com/XiangyunHuang/bookdown-broken/blob/master/index.Rmd}

\begin{CJK}{UTF8}{bsmi}
\begin{figure}
\includegraphics[width=0.65\linewidth]{202401260003-plot_files/figure-latex/skills-1} \caption{《现代统计图形》的技能栈}\label{fig:skills}
\end{figure}
\end{CJK}

\begin{figure}
\includegraphics[width=0.75\linewidth]{202401260003-plot_files/figure-latex/unnamed-chunk-64-1} \caption{cube rotate}\label{fig:unnamed-chunk-64}
\end{figure}

\hypertarget{xy-pic}{%
\chapter*{xy-pic}\label{xy-pic}}
\addcontentsline{toc}{chapter}{xy-pic}

\url{https://bookdown.org/yihui/rmarkdown-cookbook/install-latex-pkgs.html}

\texttt{tinytex::install\_tinytex()}

the following xymatrix from LaTeX package xy for xy-pic is not shown or rendered in HTML:

\texttt{\$\textbackslash{}LaTeX\$} can only be used in HTML, not PDF

\xymatrix{U\ar[ddr]_{\psi}\ar[drr]^{\varphi}\ar[dr]|-{(x,y)}\\
 & X\times_{Z}Y\ar[d]^{q}\ar[r]_{p} & X\ar[d]_{f}\\
 & Y\ar[r]^{g} & Z
}

\[
\xymatrix{U\ar[ddr]_{\psi}\ar[drr]^{\varphi}\ar[dr]|-{(x,y)}\\
 & X\times_{Z}Y\ar[d]^{q}\ar[r]_{p} & X\ar[d]_{f}\\
 & Y\ar[r]^{g} & Z
}
\]

\hypertarget{part-by-date}{%
\part{by date}\label{part-by-date}}

\hypertarget{by-date}{%
\chapter{by date}\label{by-date}}

\hypertarget{partition-1}{%
\chapter*{partition}\label{partition-1}}
\addcontentsline{toc}{chapter}{partition}

\begin{align*}
 & \left\{ A_{i}\right\} _{i\in I}=\left\{ A_{i}\middle|i\in I\right\} \text{ is a partition of a set }A\\
\Leftrightarrow & \begin{cases}
\forall i\in I\left(A_{i}\ne\emptyset\right)\\
A=\bigcup\limits _{i\in I}A_{i}\\
\forall i,j\in I\left(i\ne j\Rightarrow A_{i}\cap A_{j}=\emptyset\right)
\end{cases}
\end{align*}

\url{https://proofwiki.org/wiki/Definition:Set_Partition}

\hypertarget{section}{%
\chapter*{202401281000}\label{section}}
\addcontentsline{toc}{chapter}{202401281000}

\hypertarget{equivalence-class-1}{%
\chapter*{equivalence class}\label{equivalence-class-1}}
\addcontentsline{toc}{chapter}{equivalence class}

\begin{align*}
 & C\text{ is an equivalence class of }a\text{ on }A\\
\Leftrightarrow & \left[a\right]_{\sim}=C=\left\{ x\middle|\begin{cases}
a\in A\\
x\in A\\
x\sim a\\
\sim\text{ is an equivalence relation over }A\times A=A^{2}
\end{cases}\right\} \subseteq A\ne\emptyset\\
\Leftrightarrow & \left[a\right]=\left[a\right]_{\sim}=\left\{ x\middle|\begin{cases}
a\in A\\
x\in A\\
x\sim a\\
\sim\text{ is an equivalence relation on }A
\end{cases}\right\} \subseteq A\ne\emptyset\\
\Rightarrow & \left[a\right]_{\sim}=\left\{ x\middle|x\sim a\right\} \subseteq A\ne\emptyset
\end{align*}

where the definition of \protect\hyperlink{equivalence-relation}{equivalence relation} can be found in \ref{equivalence-relation}.

\protect\hyperlink{number-and-reference-equations}{number and reference equations}

\eqref{eq:eqclass}

\eqref{eq:emc}

\ref{thm:pyth}

\hypertarget{equivalence-relation-1}{%
\chapter*{equivalence relation}\label{equivalence-relation-1}}
\addcontentsline{toc}{chapter}{equivalence relation}

\begin{CJK}{UTF8}{bsmi}等價關係 equivalence relation \label{def:equivalence-relation}
\end{CJK}
\begin{CJK}{UTF8}{bsmi}
\begin{align*}
 & R\text{ is an equivalence relation over }A\times B\\
\Leftrightarrow & \begin{cases}
R=\sim=\left\{ \left\langle x,y\right\rangle \middle|x\sim y\right\} \subseteq A\times B & \left(\text{e}\right)\text{equivalence 等價}\\
\vdots & \vdots
\end{cases}\\
\Leftrightarrow & \begin{cases}
R=\left\{ \left\langle x,y\right\rangle \middle|xRy\right\} \subseteq A\times B & \left(R\right)\text{relation}\\
\forall\left\langle x,y\right\rangle \in R\left(xRx\right) & \left(r\right)\text{reflexive}\\
\forall\left\langle x,y\right\rangle \in R\left(xRy\Rightarrow yRx\right) & \left(s\right)\text{symmetric}\\
\forall\left\langle x,y\right\rangle ,\left\langle y,z\right\rangle \in R\left(\begin{cases}
xRy\\
yRz
\end{cases}\Rightarrow xRz\right) & \left(t\right)\text{transitive}
\end{cases}\Leftrightarrow\begin{cases}
R=\left\{ \left\langle x,y\right\rangle \middle|xRy\right\} \subseteq A\times B & \text{關係}\\
\forall\left\langle x,y\right\rangle \in R\left(\left\langle x,x\right\rangle \in R\right) & \text{自反}\\
\forall\left\langle x,y\right\rangle \in R\left(\left\langle y,x\right\rangle \in R\right) & \text{對稱}\\
\forall\left\langle x,y\right\rangle ,\left\langle y,z\right\rangle \in R\left(\left\langle x,z\right\rangle \in R\right) & \text{遞移}
\end{cases}
\end{align*}
\end{CJK}

\hypertarget{python}{%
\chapter{Python}\label{python}}

\url{https://bookdown.org/yihui/rmarkdown/language-engines.html}

\begin{Shaded}
\begin{Highlighting}[]
\FunctionTok{names}\NormalTok{(knitr}\SpecialCharTok{::}\NormalTok{knit\_engines}\SpecialCharTok{$}\FunctionTok{get}\NormalTok{())}
\end{Highlighting}
\end{Shaded}

\begin{verbatim}
##  [1] "awk"         "bash"        "coffee"      "gawk"        "groovy"     
##  [6] "haskell"     "lein"        "mysql"       "node"        "octave"     
## [11] "perl"        "php"         "psql"        "Rscript"     "ruby"       
## [16] "sas"         "scala"       "sed"         "sh"          "stata"      
## [21] "zsh"         "asis"        "asy"         "block"       "block2"     
## [26] "bslib"       "c"           "cat"         "cc"          "comment"    
## [31] "css"         "ditaa"       "dot"         "embed"       "eviews"     
## [36] "exec"        "fortran"     "fortran95"   "go"          "highlight"  
## [41] "js"          "julia"       "python"      "R"           "Rcpp"       
## [46] "sass"        "scss"        "sql"         "stan"        "targets"    
## [51] "tikz"        "verbatim"    "theorem"     "lemma"       "corollary"  
## [56] "proposition" "conjecture"  "definition"  "example"     "exercise"   
## [61] "hypothesis"  "proof"       "remark"      "solution"    "glue"       
## [66] "glue_sql"    "gluesql"
\end{verbatim}

\url{https://rstudio.github.io/reticulate/articles/python_packages.html}

\begin{Shaded}
\begin{Highlighting}[]
\NormalTok{x }\OperatorTok{=} \StringTok{\textquotesingle{}hello, python world!\textquotesingle{}}
\BuiltInTok{print}\NormalTok{(x.split(}\StringTok{\textquotesingle{} \textquotesingle{}}\NormalTok{))}
\end{Highlighting}
\end{Shaded}

\begin{verbatim}
## ['hello,', 'python', 'world!']
\end{verbatim}

\begin{Shaded}
\begin{Highlighting}[]
\FunctionTok{library}\NormalTok{(reticulate)}
\end{Highlighting}
\end{Shaded}

\begin{verbatim}
## Warning: package 'reticulate' was built under R version 4.2.3
\end{verbatim}

\begin{Shaded}
\begin{Highlighting}[]
\FunctionTok{virtualenv\_python}\NormalTok{()}
\end{Highlighting}
\end{Shaded}

\begin{verbatim}
## [1] "D:/Users/115381/Documents/.virtualenvs/r-reticulate/Scripts/python.exe"
\end{verbatim}

\begin{Shaded}
\begin{Highlighting}[]
\FunctionTok{library}\NormalTok{(reticulate)}
\FunctionTok{conda\_list}\NormalTok{()}
\end{Highlighting}
\end{Shaded}

\begin{verbatim}
##                name                                            python
## 1              base                          D:\\Anaconda3/python.exe
## 2          fiftyone          D:\\Anaconda3\\envs\\fiftyone/python.exe
## 3             keras             D:\\Anaconda3\\envs\\keras/python.exe
## 4           labelme           D:\\Anaconda3\\envs\\labelme/python.exe
## 5             manim             D:\\Anaconda3\\envs\\manim/python.exe
## 6            mmyolo            D:\\Anaconda3\\envs\\mmyolo/python.exe
## 7 rsconnect-jupyter D:\\Anaconda3\\envs\\rsconnect-jupyter/python.exe
## 8           sandbox           D:\\Anaconda3\\envs\\sandbox/python.exe
## 9       sandbox-3.9       D:\\Anaconda3\\envs\\sandbox-3.9/python.exe
\end{verbatim}

\begin{Shaded}
\begin{Highlighting}[]
\FunctionTok{library}\NormalTok{(reticulate)}
\FunctionTok{virtualenv\_list}\NormalTok{()}
\end{Highlighting}
\end{Shaded}

\begin{verbatim}
## [1] "r-reticulate"
\end{verbatim}

\url{https://rstudio.github.io/reticulate/reference/install_python.html}

\begin{Shaded}
\begin{Highlighting}[]
\FunctionTok{library}\NormalTok{(reticulate)}
\NormalTok{version }\OtherTok{\textless{}{-}} \StringTok{"3.9.12"}
\CommentTok{\# install\_python(version)}

\CommentTok{\# create a new environment}
\CommentTok{\# virtualenv\_create("r{-}reticulate", version = version)}

\CommentTok{\# use\_virtualenv("r{-}reticulate")}

\CommentTok{\# install MatPlotLib}
\CommentTok{\# virtualenv\_install("r{-}reticulate", "matplotlib")}

\CommentTok{\# import MatPlotLib (it will be automatically discovered in "r{-}reticulate")}
\NormalTok{matplotlib }\OtherTok{\textless{}{-}} \FunctionTok{import}\NormalTok{(}\StringTok{"matplotlib"}\NormalTok{)}
\end{Highlighting}
\end{Shaded}

copy \texttt{C:\textbackslash{}Users\textbackslash{}RW\textbackslash{}AppData\textbackslash{}Local\textbackslash{}r-reticulate\textbackslash{}r-reticulate\textbackslash{}pyenv\textbackslash{}pyenv-win\textbackslash{}versions\textbackslash{}3.9.12\textbackslash{}tcl\textbackslash{}tcl8.6} and \texttt{C:\textbackslash{}Users\textbackslash{}RW\textbackslash{}AppData\textbackslash{}Local\textbackslash{}r-reticulate\textbackslash{}r-reticulate\textbackslash{}pyenv\textbackslash{}pyenv-win\textbackslash{}versions\textbackslash{}3.9.12\textbackslash{}tcl\textbackslash{}tk8.6} two folders to the folder \texttt{C:\textbackslash{}Users\textbackslash{}RW\textbackslash{}AppData\textbackslash{}Local\textbackslash{}r-reticulate\textbackslash{}r-reticulate\textbackslash{}pyenv\textbackslash{}pyenv-win\textbackslash{}versions\textbackslash{}3.9.12\textbackslash{}Lib}

\begin{Shaded}
\begin{Highlighting}[]
\CommentTok{\# library(reticulate)}
\CommentTok{\# use\_virtualenv("r{-}reticulate")}
\CommentTok{\# \# matplotlib \textless{}{-} import("matplotlib")}
\CommentTok{\# matplotlib$use("Agg", force = TRUE)}
\end{Highlighting}
\end{Shaded}

\begin{Shaded}
\begin{Highlighting}[]
\ImportTok{import}\NormalTok{ matplotlib.pyplot }\ImportTok{as}\NormalTok{ plt}
\NormalTok{plt.plot([}\DecValTok{0}\NormalTok{, }\DecValTok{2}\NormalTok{, }\DecValTok{1}\NormalTok{, }\DecValTok{4}\NormalTok{])}
\NormalTok{plt.show()}
\end{Highlighting}
\end{Shaded}

\includegraphics{202401292317-Python_files/figure-latex/unnamed-chunk-8-1.pdf}

\hypertarget{tikz-1}{%
\chapter*{TiKZ}\label{tikz-1}}
\addcontentsline{toc}{chapter}{TiKZ}

TiKZ and PGFplots

What's the relation between packages PGFplots and TikZ?

\url{https://tex.stackexchange.com/questions/285925/whats-the-relation-between-packages-pgfplots-and-tikz}

\url{https://www.youtube.com/watch?v=bQugbYq0BVA}

\url{https://www.youtube.com/watch?v=ft4Kg9emK1k\&list=PLg5nrpKdkk2DWcg3scb75AknF7DJXs8lk\&index=18}

\begin{Shaded}
\begin{Highlighting}[]
\KeywordTok{\textbackslash{}begin}\NormalTok{\{}\ExtensionTok{tikzpicture}\NormalTok{\}}
  \FunctionTok{\textbackslash{}def\textbackslash{}a}\NormalTok{\{1.5\} }\CommentTok{\% amplitude}
  \FunctionTok{\textbackslash{}def\textbackslash{}b}\NormalTok{\{2\}   }\CommentTok{\% frequency}
  \FunctionTok{\textbackslash{}draw}\NormalTok{[{-}\textgreater{}] ({-}0.2,0) {-}{-} (4.2,0) node[right, font=}\FunctionTok{\textbackslash{}small}\NormalTok{] \{}\SpecialStringTok{$x$}\NormalTok{\};}
  \FunctionTok{\textbackslash{}draw}\NormalTok{[{-}\textgreater{}] (0,{-}4) {-}{-} (0,0.5) node[above] \{}\SpecialStringTok{$y$}\NormalTok{\};}
  \FunctionTok{\textbackslash{}draw}\NormalTok{[domain=0:4,smooth,variable=}\FunctionTok{\textbackslash{}t}\NormalTok{,blue,thick] }
\NormalTok{    plot (\{}\FunctionTok{\textbackslash{}a}\NormalTok{ * (}\FunctionTok{\textbackslash{}b*\textbackslash{}t}\NormalTok{ {-} sin(deg(}\FunctionTok{\textbackslash{}b*\textbackslash{}t}\NormalTok{)))\},\{{-}}\FunctionTok{\textbackslash{}a}\NormalTok{ * (1 {-} cos(deg(}\FunctionTok{\textbackslash{}b*\textbackslash{}t}\NormalTok{)))\});}
  \CommentTok{\% \textbackslash{}node[above] at (2, 0.5) \{Brachistochrone Curve\};}
  \FunctionTok{\textbackslash{}node}\NormalTok{[above, font=}\FunctionTok{\textbackslash{}footnotesize}\NormalTok{] at (2, 1) \{Brachistochrone Curve\};}
  \FunctionTok{\textbackslash{}node}\NormalTok{[above, font=}\FunctionTok{\textbackslash{}footnotesize}\NormalTok{] at (2, 0) \{}\SpecialStringTok{$}\KeywordTok{\textbackslash{}begin}\NormalTok{\{}\ExtensionTok{aligned}\NormalTok{\}}
\SpecialStringTok{\& x=r(t{-}}\SpecialCharTok{\textbackslash{}sin}\SpecialStringTok{ t) }\SpecialCharTok{\textbackslash{}\textbackslash{}}
\SpecialStringTok{\& y=r(1{-}}\SpecialCharTok{\textbackslash{}cos}\SpecialStringTok{ t)}
\KeywordTok{\textbackslash{}end}\NormalTok{\{}\ExtensionTok{aligned}\NormalTok{\}}\SpecialStringTok{$}\NormalTok{\};}
\KeywordTok{\textbackslash{}end}\NormalTok{\{}\ExtensionTok{tikzpicture}\NormalTok{\}}
\end{Highlighting}
\end{Shaded}

\begin{figure}
\includegraphics[width=0.9\linewidth]{202401311000-TiKZ_files/figure-latex/unnamed-chunk-2-1} \caption{Brachistochrone Curve}\label{fig:unnamed-chunk-2}
\end{figure}

\begin{figure}
\includegraphics[width=0.9\linewidth]{202401311000-TiKZ_files/figure-latex/unnamed-chunk-3-1} \caption{Brachistochrone Curve}\label{fig:unnamed-chunk-3}
\end{figure}

\url{https://zhuanlan.zhihu.com/p/127155579?utm_psn=1741479950987960320}

1

\begin{Shaded}
\begin{Highlighting}[]
\KeywordTok{\textbackslash{}begin}\NormalTok{\{}\ExtensionTok{tikzpicture}\NormalTok{\}}
  \FunctionTok{\textbackslash{}draw}\NormalTok{ ({-}1,1){-}{-}(0,0){-}{-}(1,2);}
\KeywordTok{\textbackslash{}end}\NormalTok{\{}\ExtensionTok{tikzpicture}\NormalTok{\}}
\end{Highlighting}
\end{Shaded}

\begin{figure}
\includegraphics[width=0.5\linewidth]{202401311000-TiKZ_files/figure-latex/unnamed-chunk-5-1} \end{figure}

\begin{figure}
\includegraphics[width=0.5\linewidth]{202401311000-TiKZ_files/figure-latex/unnamed-chunk-6-1} \end{figure}

\begin{figure}
\includegraphics[width=1\linewidth]{202401311000-TiKZ_files/figure-latex/unnamed-chunk-7-1} \end{figure}

2

\begin{figure}
\includegraphics[width=0.9\linewidth]{202401311000-TiKZ_files/figure-latex/unnamed-chunk-8-1} \end{figure}

3

\begin{figure}
\includegraphics[width=0.25\linewidth]{202401311000-TiKZ_files/figure-latex/unnamed-chunk-9-1} \end{figure}

\begin{Shaded}
\begin{Highlighting}[]
\KeywordTok{\textbackslash{}begin}\NormalTok{\{}\ExtensionTok{tikzpicture}\NormalTok{\}}
  \FunctionTok{\textbackslash{}draw}\NormalTok{[rounded corners] ({-}1,1){-}{-}(0,0){-}{-}(1,2){-}{-}({-}1,1);}
\KeywordTok{\textbackslash{}end}\NormalTok{\{}\ExtensionTok{tikzpicture}\NormalTok{\}}
\end{Highlighting}
\end{Shaded}

\begin{figure}
\includegraphics[width=0.25\linewidth]{202401311000-TiKZ_files/figure-latex/unnamed-chunk-11-1} \caption{rounded corner pseudo-closed triangle}\label{fig:unnamed-chunk-11}
\end{figure}

\begin{Shaded}
\begin{Highlighting}[]
\KeywordTok{\textbackslash{}begin}\NormalTok{\{}\ExtensionTok{tikzpicture}\NormalTok{\}}
  \FunctionTok{\textbackslash{}draw}\NormalTok{[rounded corners] ({-}1,1){-}{-}(0,0){-}{-}(1,2){-}{-}cycle;}
\KeywordTok{\textbackslash{}end}\NormalTok{\{}\ExtensionTok{tikzpicture}\NormalTok{\}}
\end{Highlighting}
\end{Shaded}

\begin{figure}
\includegraphics[width=0.25\linewidth]{202401311000-TiKZ_files/figure-latex/unnamed-chunk-13-1} \caption{rounded corner triangle}\label{fig:unnamed-chunk-13}
\end{figure}

\begin{figure}
\includegraphics[width=0.25\linewidth]{202401311000-TiKZ_files/figure-latex/unnamed-chunk-14-1} \caption{triangle vs. pseudo-closed triangle}\label{fig:unnamed-chunk-14}
\end{figure}

\begin{Shaded}
\begin{Highlighting}[]
\KeywordTok{\textbackslash{}begin}\NormalTok{\{}\ExtensionTok{tikzpicture}\NormalTok{\}}
  \FunctionTok{\textbackslash{}draw}\NormalTok{ (0,0) rectangle (4,2);}
\KeywordTok{\textbackslash{}end}\NormalTok{\{}\ExtensionTok{tikzpicture}\NormalTok{\}}
\end{Highlighting}
\end{Shaded}

\begin{figure}
\includegraphics[width=0.25\linewidth]{202401311000-TiKZ_files/figure-latex/unnamed-chunk-16-1} \caption{rectangle}\label{fig:unnamed-chunk-16}
\end{figure}

\begin{Shaded}
\begin{Highlighting}[]
\KeywordTok{\textbackslash{}begin}\NormalTok{\{}\ExtensionTok{tikzpicture}\NormalTok{\}}
  \FunctionTok{\textbackslash{}draw}\NormalTok{ (0,0) rectangle (2,2);}
\KeywordTok{\textbackslash{}end}\NormalTok{\{}\ExtensionTok{tikzpicture}\NormalTok{\}}
\end{Highlighting}
\end{Shaded}

\begin{figure}
\includegraphics[width=0.25\linewidth]{202401311000-TiKZ_files/figure-latex/unnamed-chunk-18-1} \caption{square}\label{fig:unnamed-chunk-18}
\end{figure}

\begin{Shaded}
\begin{Highlighting}[]
\KeywordTok{\textbackslash{}begin}\NormalTok{\{}\ExtensionTok{tikzpicture}\NormalTok{\}}
  \FunctionTok{\textbackslash{}draw}\NormalTok{ (0,0) circle (1);}
\KeywordTok{\textbackslash{}end}\NormalTok{\{}\ExtensionTok{tikzpicture}\NormalTok{\}}
\end{Highlighting}
\end{Shaded}

\begin{figure}
\includegraphics[width=0.25\linewidth]{202401311000-TiKZ_files/figure-latex/unnamed-chunk-20-1} \caption{circle}\label{fig:unnamed-chunk-20}
\end{figure}

\begin{Shaded}
\begin{Highlighting}[]
\KeywordTok{\textbackslash{}begin}\NormalTok{\{}\ExtensionTok{tikzpicture}\NormalTok{\}}
  \FunctionTok{\textbackslash{}draw}\NormalTok{ (0,0) circle (1);}
  \FunctionTok{\textbackslash{}draw}\NormalTok{ (0,0) rectangle (2,2);}
\KeywordTok{\textbackslash{}end}\NormalTok{\{}\ExtensionTok{tikzpicture}\NormalTok{\}}
\end{Highlighting}
\end{Shaded}

\begin{figure}
\includegraphics[width=0.25\linewidth]{202401311000-TiKZ_files/figure-latex/unnamed-chunk-22-1} \caption{circle and square}\label{fig:unnamed-chunk-22}
\end{figure}

\begin{Shaded}
\begin{Highlighting}[]
\KeywordTok{\textbackslash{}begin}\NormalTok{\{}\ExtensionTok{tikzpicture}\NormalTok{\}}
  \FunctionTok{\textbackslash{}draw}\NormalTok{ (1,1) ellipse (2 and 1);}
\KeywordTok{\textbackslash{}end}\NormalTok{\{}\ExtensionTok{tikzpicture}\NormalTok{\}}
\end{Highlighting}
\end{Shaded}

\begin{figure}
\includegraphics[width=0.25\linewidth]{202401311000-TiKZ_files/figure-latex/unnamed-chunk-24-1} \caption{ellipse}\label{fig:unnamed-chunk-24}
\end{figure}

\begin{Shaded}
\begin{Highlighting}[]
\KeywordTok{\textbackslash{}begin}\NormalTok{\{}\ExtensionTok{tikzpicture}\NormalTok{\}}
  \FunctionTok{\textbackslash{}draw}\NormalTok{ (1 ,1) arc (0:270:1);}
  \FunctionTok{\textbackslash{}draw}\NormalTok{ (6 ,1) arc (0:270:2 and 1);}
\KeywordTok{\textbackslash{}end}\NormalTok{\{}\ExtensionTok{tikzpicture}\NormalTok{\}}
\end{Highlighting}
\end{Shaded}

\begin{figure}
\includegraphics[width=0.25\linewidth]{202401311000-TiKZ_files/figure-latex/unnamed-chunk-26-1} \caption{circle and ellipse arcs}\label{fig:unnamed-chunk-26}
\end{figure}

\begin{Shaded}
\begin{Highlighting}[]
\KeywordTok{\textbackslash{}begin}\NormalTok{\{}\ExtensionTok{tikzpicture}\NormalTok{\}}
  \FunctionTok{\textbackslash{}draw}\NormalTok{ ({-}1,1) parabola bend (0,0) (2,4);}
\KeywordTok{\textbackslash{}end}\NormalTok{\{}\ExtensionTok{tikzpicture}\NormalTok{\}}
\end{Highlighting}
\end{Shaded}

\begin{figure}
\includegraphics[width=0.25\linewidth]{202401311000-TiKZ_files/figure-latex/unnamed-chunk-28-1} \caption{parabola arc}\label{fig:unnamed-chunk-28}
\end{figure}

\begin{Shaded}
\begin{Highlighting}[]
\KeywordTok{\textbackslash{}begin}\NormalTok{\{}\ExtensionTok{tikzpicture}\NormalTok{\}}
  \FunctionTok{\textbackslash{}draw}\NormalTok{ ({-}1,1) parabola bend (0,0) (2,4);}
  \FunctionTok{\textbackslash{}filldraw}
\NormalTok{    ({-}1,1) circle (.05)}
\NormalTok{    ( 0,0) circle (.05)}
\NormalTok{    ( 1,1) circle (.05)}
\NormalTok{    ( 2,4) circle (.05);}
\KeywordTok{\textbackslash{}end}\NormalTok{\{}\ExtensionTok{tikzpicture}\NormalTok{\}}
\end{Highlighting}
\end{Shaded}

\begin{figure}
\includegraphics[width=0.25\linewidth]{202401311000-TiKZ_files/figure-latex/unnamed-chunk-30-1} \caption{parabola arc with points}\label{fig:unnamed-chunk-30}
\end{figure}

\begin{Shaded}
\begin{Highlighting}[]
\KeywordTok{\textbackslash{}begin}\NormalTok{\{}\ExtensionTok{tikzpicture}\NormalTok{\}}
  \FunctionTok{\textbackslash{}draw}\NormalTok{ [step=20pt] (0,0) grid (3,2);}
  \FunctionTok{\textbackslash{}draw}\NormalTok{ [help lines ,step=20pt] (4,0) grid (7,2);}
\KeywordTok{\textbackslash{}end}\NormalTok{\{}\ExtensionTok{tikzpicture}\NormalTok{\}}
\end{Highlighting}
\end{Shaded}

\begin{figure}
\includegraphics[width=0.75\linewidth]{202401311000-TiKZ_files/figure-latex/unnamed-chunk-32-1} \caption{grid and help lines}\label{fig:unnamed-chunk-32}
\end{figure}

\begin{figure}
\includegraphics[width=0.75\linewidth]{202401311000-TiKZ_files/figure-latex/unnamed-chunk-33-1} \caption{grid and help lines}\label{fig:unnamed-chunk-33}
\end{figure}

\begin{Shaded}
\begin{Highlighting}[]
\KeywordTok{\textbackslash{}begin}\NormalTok{\{}\ExtensionTok{tikzpicture}\NormalTok{\}[scale=0.25]}
  \FunctionTok{\textbackslash{}draw}\NormalTok{ [{-}\textgreater{}] (0,0){-}{-}(9,0);}
  \FunctionTok{\textbackslash{}draw}\NormalTok{ [\textless{}{-}] (0,1){-}{-}(9,1);}
  \FunctionTok{\textbackslash{}draw}\NormalTok{ [\textless{}{-}\textgreater{}] (0,2){-}{-}(9,2);}
  \FunctionTok{\textbackslash{}draw}\NormalTok{ [\textgreater{}{-}\textgreater{}\textgreater{}] (0,3){-}{-}(9,3);}
  \FunctionTok{\textbackslash{}draw}\NormalTok{ [|\textless{}{-}\textgreater{}|] (0,4){-}{-}(9,4);}
\KeywordTok{\textbackslash{}end}\NormalTok{\{}\ExtensionTok{tikzpicture}\NormalTok{\}}
\end{Highlighting}
\end{Shaded}

\begin{figure}
\includegraphics[width=0.75\linewidth]{202401311000-TiKZ_files/figure-latex/unnamed-chunk-35-1} \caption{arrows}\label{fig:unnamed-chunk-35}
\end{figure}

\begin{Shaded}
\begin{Highlighting}[]
\KeywordTok{\textbackslash{}begin}\NormalTok{\{}\ExtensionTok{tikzpicture}\NormalTok{\}}
  \FunctionTok{\textbackslash{}draw}\NormalTok{ [line width =2pt] (0,6){-}{-}(9,6); }
  \FunctionTok{\textbackslash{}draw}\NormalTok{ [dotted]          (0,5){-}{-}(9,5); }
  \FunctionTok{\textbackslash{}draw}\NormalTok{ [densely dotted]  (0,4){-}{-}(9,4); }
  \FunctionTok{\textbackslash{}draw}\NormalTok{ [loosely dotted]  (0,3){-}{-}(9,3); }
  \FunctionTok{\textbackslash{}draw}\NormalTok{ [dashed]          (0,2){-}{-}(9,2); }
  \FunctionTok{\textbackslash{}draw}\NormalTok{ [densely dashed]  (0,1){-}{-}(9,1); }
  \FunctionTok{\textbackslash{}draw}\NormalTok{ [loosely dashed]  (0,0){-}{-}(9,0);}
\KeywordTok{\textbackslash{}end}\NormalTok{\{}\ExtensionTok{tikzpicture}\NormalTok{\}}
\end{Highlighting}
\end{Shaded}

\begin{figure}
\includegraphics[width=0.75\linewidth]{202401311000-TiKZ_files/figure-latex/unnamed-chunk-37-1} \caption{arrows}\label{fig:unnamed-chunk-37}
\end{figure}

\begin{Shaded}
\begin{Highlighting}[]
\KeywordTok{\textbackslash{}begin}\NormalTok{\{}\ExtensionTok{tikzpicture}\NormalTok{\}[dline/.style=\{color= blue, line width=2pt\}]}
  \FunctionTok{\textbackslash{}draw}\NormalTok{[dline] (0,0){-}{-}(9,0); }
\KeywordTok{\textbackslash{}end}\NormalTok{\{}\ExtensionTok{tikzpicture}\NormalTok{\}}
\end{Highlighting}
\end{Shaded}

\begin{figure}
\includegraphics[width=0.75\linewidth]{202401311000-TiKZ_files/figure-latex/unnamed-chunk-39-1} \caption{head styling}\label{fig:unnamed-chunk-39}
\end{figure}

\begin{Shaded}
\begin{Highlighting}[]
\KeywordTok{\textbackslash{}begin}\NormalTok{\{}\ExtensionTok{tikzpicture}\NormalTok{\}}
  \FunctionTok{\textbackslash{}draw}\NormalTok{ (0,0) rectangle (2,2);}
  \FunctionTok{\textbackslash{}draw}\NormalTok{[shift=\{( 3, 0)\}] (0,0) rectangle (2,2);}
  \FunctionTok{\textbackslash{}draw}\NormalTok{[shift=\{( 0, 3)\}] (0,0) rectangle (2,2);}
  \FunctionTok{\textbackslash{}draw}\NormalTok{[shift=\{( 0,{-}3)\}] (0,0) rectangle (2,2);}
  \FunctionTok{\textbackslash{}draw}\NormalTok{[shift=\{({-}3, 0)\}] (0,0) rectangle (2,2);}
  \FunctionTok{\textbackslash{}draw}\NormalTok{[shift=\{( 3, 3)\}] (0,0) rectangle (2,2);}
  \FunctionTok{\textbackslash{}draw}\NormalTok{[shift=\{({-}3, 3)\}] (0,0) rectangle (2,2);}
  \FunctionTok{\textbackslash{}draw}\NormalTok{[shift=\{( 3,{-}3)\}] (0,0) rectangle (2,2);}
  \FunctionTok{\textbackslash{}draw}\NormalTok{[shift=\{({-}3,{-}3)\}] (0,0) rectangle (2,2);}
\KeywordTok{\textbackslash{}end}\NormalTok{\{}\ExtensionTok{tikzpicture}\NormalTok{\}}
\end{Highlighting}
\end{Shaded}

\begin{figure}
\includegraphics[width=0.75\linewidth]{202401311000-TiKZ_files/figure-latex/unnamed-chunk-41-1} \caption{transform: shift}\label{fig:unnamed-chunk-41}
\end{figure}

\begin{Shaded}
\begin{Highlighting}[]
\KeywordTok{\textbackslash{}begin}\NormalTok{\{}\ExtensionTok{tikzpicture}\NormalTok{\}}
  \FunctionTok{\textbackslash{}draw}\NormalTok{ (0,0) rectangle (2,2);}
  \FunctionTok{\textbackslash{}draw}\NormalTok{[xshift= 100pt] (0,0) rectangle (2,2);}
  \FunctionTok{\textbackslash{}draw}\NormalTok{[xshift={-}100pt] (0,0) rectangle (2,2);}
  \FunctionTok{\textbackslash{}draw}\NormalTok{[yshift= 100pt] (0,0) rectangle (2,2);}
  \FunctionTok{\textbackslash{}draw}\NormalTok{[yshift={-}100pt] (0,0) rectangle (2,2);}
\KeywordTok{\textbackslash{}end}\NormalTok{\{}\ExtensionTok{tikzpicture}\NormalTok{\}}
\end{Highlighting}
\end{Shaded}

\begin{figure}
\includegraphics[width=0.75\linewidth]{202401311000-TiKZ_files/figure-latex/unnamed-chunk-43-1} \caption{transform: shift x, y}\label{fig:unnamed-chunk-43}
\end{figure}

\begin{Shaded}
\begin{Highlighting}[]
\KeywordTok{\textbackslash{}begin}\NormalTok{\{}\ExtensionTok{tikzpicture}\NormalTok{\}}
  \FunctionTok{\textbackslash{}draw}\NormalTok{ (0,0) rectangle (2,2);}
  \FunctionTok{\textbackslash{}draw}\NormalTok{[xshift= 100pt, xscale=1.5] (0,0) rectangle (2,2);}
  \FunctionTok{\textbackslash{}draw}\NormalTok{[yshift= 100pt, xscale=0.5] (0,0) rectangle (2,2);}
  \FunctionTok{\textbackslash{}draw}\NormalTok{[xshift={-}100pt, yscale=1.5] (0,0) rectangle (2,2);}
  \FunctionTok{\textbackslash{}draw}\NormalTok{[yshift={-}100pt, yscale=0.5] (0,0) rectangle (2,2);}
\KeywordTok{\textbackslash{}end}\NormalTok{\{}\ExtensionTok{tikzpicture}\NormalTok{\}}
\end{Highlighting}
\end{Shaded}

\begin{figure}
\includegraphics[width=0.75\linewidth]{202401311000-TiKZ_files/figure-latex/unnamed-chunk-45-1} \caption{transform: scale x, y}\label{fig:unnamed-chunk-45}
\end{figure}

\begin{Shaded}
\begin{Highlighting}[]
\KeywordTok{\textbackslash{}begin}\NormalTok{\{}\ExtensionTok{tikzpicture}\NormalTok{\}}
  \FunctionTok{\textbackslash{}draw}\NormalTok{ (0,0) rectangle (2,2);}
  \FunctionTok{\textbackslash{}draw}\NormalTok{[xshift= 100pt, xscale=1.5] (0,0) rectangle (2,2);}
  \FunctionTok{\textbackslash{}draw}\NormalTok{[yshift= 100pt, yscale=1.5] (0,0) rectangle (2,2);}
  \FunctionTok{\textbackslash{}draw}\NormalTok{[xshift={-}100pt, xscale=0.5] (0,0) rectangle (2,2);}
  \FunctionTok{\textbackslash{}draw}\NormalTok{[yshift={-}100pt, yscale=0.5] (0,0) rectangle (2,2);}
\KeywordTok{\textbackslash{}end}\NormalTok{\{}\ExtensionTok{tikzpicture}\NormalTok{\}}
\end{Highlighting}
\end{Shaded}

\begin{figure}
\includegraphics[width=0.75\linewidth]{202401311000-TiKZ_files/figure-latex/unnamed-chunk-47-1} \caption{transform: scale}\label{fig:unnamed-chunk-47}
\end{figure}

\begin{Shaded}
\begin{Highlighting}[]
\KeywordTok{\textbackslash{}begin}\NormalTok{\{}\ExtensionTok{tikzpicture}\NormalTok{\}}
  \FunctionTok{\textbackslash{}draw}\NormalTok{ (0,0) rectangle (2,2);}
  \FunctionTok{\textbackslash{}draw}\NormalTok{[xshift=125pt,rotate=45] (0,0) rectangle (2,2);}
  \FunctionTok{\textbackslash{}draw}\NormalTok{[xshift=175pt,rotate around=\{45:(2 ,2)\}] (0,0) rectangle (2,2);}
\KeywordTok{\textbackslash{}end}\NormalTok{\{}\ExtensionTok{tikzpicture}\NormalTok{\}}
\end{Highlighting}
\end{Shaded}

\begin{figure}
\includegraphics[width=0.75\linewidth]{202401311000-TiKZ_files/figure-latex/unnamed-chunk-49-1} \caption{transform: rotate}\label{fig:unnamed-chunk-49}
\end{figure}

\begin{Shaded}
\begin{Highlighting}[]
\KeywordTok{\textbackslash{}begin}\NormalTok{\{}\ExtensionTok{tikzpicture}\NormalTok{\}}
  \FunctionTok{\textbackslash{}draw}\NormalTok{ (0,0) rectangle (2,2);}
  \FunctionTok{\textbackslash{}draw}\NormalTok{[xshift=70pt,xslant=1] (0,0) rectangle (2,2);}
  \FunctionTok{\textbackslash{}draw}\NormalTok{[yshift=70pt,yslant=1] (0,0) rectangle (2,2);}
\KeywordTok{\textbackslash{}end}\NormalTok{\{}\ExtensionTok{tikzpicture}\NormalTok{\}}
\end{Highlighting}
\end{Shaded}

\begin{figure}
\includegraphics[width=0.75\linewidth]{202401311000-TiKZ_files/figure-latex/unnamed-chunk-51-1} \caption{transform: slant}\label{fig:unnamed-chunk-51}
\end{figure}

\begin{Shaded}
\begin{Highlighting}[]
\FunctionTok{\textbackslash{}tikzset}\NormalTok{\{}
\NormalTok{  box/.style=\{}
\NormalTok{    draw=blue,}
\NormalTok{    rectangle,}
\NormalTok{    rounded corners=5pt,}
\NormalTok{    minimum width=50pt,}
\NormalTok{    minimum height=20pt,}
\NormalTok{    inner sep=5pt}
\NormalTok{  \}}
\NormalTok{\}}
\KeywordTok{\textbackslash{}begin}\NormalTok{\{}\ExtensionTok{tikzpicture}\NormalTok{\}}
  \FunctionTok{\textbackslash{}node}\NormalTok{[box] (1) at(0,0) \{1\};}
  \FunctionTok{\textbackslash{}node}\NormalTok{[box] (2) at(4,0) \{2\};}
  \FunctionTok{\textbackslash{}node}\NormalTok{[box] (3) at(8,0) \{3\};}
  \FunctionTok{\textbackslash{}draw}\NormalTok{[{-}\textgreater{}] (1){-}{-}(2);}
  \FunctionTok{\textbackslash{}draw}\NormalTok{[{-}\textgreater{}] (2){-}{-}(3);}
  \FunctionTok{\textbackslash{}node}\NormalTok{ at(2,1) \{a\};}
  \FunctionTok{\textbackslash{}node}\NormalTok{ at(6,1) \{b\};}
\KeywordTok{\textbackslash{}end}\NormalTok{\{}\ExtensionTok{tikzpicture}\NormalTok{\}}
\end{Highlighting}
\end{Shaded}

\begin{figure}
\includegraphics[width=0.75\linewidth]{202401311000-TiKZ_files/figure-latex/unnamed-chunk-53-1} \caption{flowchart}\label{fig:unnamed-chunk-53}
\end{figure}

\begin{Shaded}
\begin{Highlighting}[]
\FunctionTok{\textbackslash{}tikzset}\NormalTok{\{}
\NormalTok{  box/.style=\{}
\NormalTok{    draw=blue,}
\NormalTok{    fill=blue!20,}
\NormalTok{    rectangle,}
\NormalTok{    rounded corners=5pt,}
\NormalTok{    minimum height=20pt,}
\NormalTok{    inner sep=5pt}
\NormalTok{  \}}
\NormalTok{\}}
\KeywordTok{\textbackslash{}begin}\NormalTok{\{}\ExtensionTok{tikzpicture}\NormalTok{\}}
  \FunctionTok{\textbackslash{}node}\NormalTok{[box] \{1\}}
\NormalTok{      child \{node[box] \{2\}\}}
\NormalTok{      child \{node[box] \{3\}}
\NormalTok{          child \{node[box] \{4\}\}}
\NormalTok{          child \{node[box] \{5\}\}}
\NormalTok{          child \{node[box] \{6\}\}}
\NormalTok{      \};}
\KeywordTok{\textbackslash{}end}\NormalTok{\{}\ExtensionTok{tikzpicture}\NormalTok{\}}
\end{Highlighting}
\end{Shaded}

\begin{figure}
\includegraphics[width=0.75\linewidth]{202401311000-TiKZ_files/figure-latex/unnamed-chunk-55-1} \caption{tree}\label{fig:unnamed-chunk-55}
\end{figure}

\begin{Shaded}
\begin{Highlighting}[]
\KeywordTok{\textbackslash{}begin}\NormalTok{\{}\ExtensionTok{tikzpicture}\NormalTok{\}}
  \FunctionTok{\textbackslash{}draw}\NormalTok{[{-}\textgreater{}] ({-}0.2,0){-}{-}(6,0) node[right] \{}\SpecialStringTok{$x$}\NormalTok{\};}
  \FunctionTok{\textbackslash{}draw}\NormalTok{[{-}\textgreater{}] (0,{-}0.2){-}{-}(0,6) node[above] \{}\SpecialStringTok{$f(x)$}\NormalTok{\};}
  \FunctionTok{\textbackslash{}draw}\NormalTok{[domain=0:4] plot (}\FunctionTok{\textbackslash{}x}\NormalTok{ ,\{0.1* exp(}\FunctionTok{\textbackslash{}x}\NormalTok{)\}) node[right] \{}\SpecialStringTok{$f(x)=}\SpecialCharTok{\textbackslash{}frac}\SpecialStringTok{\{1\}\{10\}e\^{}x$}\NormalTok{\};}
\KeywordTok{\textbackslash{}end}\NormalTok{\{}\ExtensionTok{tikzpicture}\NormalTok{\}}
\end{Highlighting}
\end{Shaded}

\begin{figure}
\includegraphics[width=0.75\linewidth]{202401311000-TiKZ_files/figure-latex/unnamed-chunk-57-1} \caption{tree}\label{fig:unnamed-chunk-57}
\end{figure}

\url{https://stackoverflow.com/questions/64897575/tikz-libraries-in-bookdown}

It turns out that you can simply put the \texttt{\textbackslash{}usetikzlibrary\{...\}} command directly before the \texttt{\textbackslash{}begin\{tikzpicture\}} and everything works fine :)

\url{https://stackoverflow.com/questions/56211210/r-markdown-document-with-html-docx-output-using-latex-package-bbm}

\url{https://tex.stackexchange.com/questions/171711/how-to-include-latex-package-in-r-markdown}

\hypertarget{d-1}{%
\section{3D}\label{d-1}}

\url{https://zhuanlan.zhihu.com/p/431732330?utm_psn=1741857547550638080}

\begin{Shaded}
\begin{Highlighting}[]
\KeywordTok{\textbackslash{}begin}\NormalTok{\{}\ExtensionTok{tikzpicture}\NormalTok{\}}
  \FunctionTok{\textbackslash{}coordinate}\NormalTok{ (A) at ( 1, 1, 1);}
  \FunctionTok{\textbackslash{}coordinate}\NormalTok{ (B) at ( 1, 1,{-}1);}
  \FunctionTok{\textbackslash{}coordinate}\NormalTok{ (C) at ( 1,{-}1,{-}1);}
  \FunctionTok{\textbackslash{}coordinate}\NormalTok{ (D) at ( 1,{-}1, 1);}
  \FunctionTok{\textbackslash{}coordinate}\NormalTok{ (E) at ({-}1,{-}1, 1);}
  \FunctionTok{\textbackslash{}coordinate}\NormalTok{ (F) at ({-}1,{-}1,{-}1);}
  \FunctionTok{\textbackslash{}coordinate}\NormalTok{ (G) at ({-}1, 1,{-}1);}
  \FunctionTok{\textbackslash{}coordinate}\NormalTok{ (H) at ({-}1, 1, 1);}
  \FunctionTok{\textbackslash{}draw}\NormalTok{ (A) node[right=1pt] \{}\SpecialStringTok{$A$}\NormalTok{\}{-}{-}}
\NormalTok{        (B) node[right=1pt] \{}\SpecialStringTok{$B$}\NormalTok{\}{-}{-}}
\NormalTok{        (C) node[right=1pt] \{}\SpecialStringTok{$C$}\NormalTok{\}{-}{-}}
\NormalTok{        (D) node[right=1pt] \{}\SpecialStringTok{$D$}\NormalTok{\}{-}{-}}
\NormalTok{        (E) node[left= 1pt] \{}\SpecialStringTok{$E$}\NormalTok{\}{-}{-}}
\NormalTok{        (F) node[right=1pt] \{}\SpecialStringTok{$F$}\NormalTok{\}{-}{-}}
\NormalTok{        (G) node[right=1pt] \{}\SpecialStringTok{$G$}\NormalTok{\}{-}{-}}
\NormalTok{        (H) node[left= 1pt] \{}\SpecialStringTok{$H$}\NormalTok{\}{-}{-}}
\NormalTok{        (A) node[right=1pt] \{}\SpecialStringTok{$A$}\NormalTok{\};}
\KeywordTok{\textbackslash{}end}\NormalTok{\{}\ExtensionTok{tikzpicture}\NormalTok{\}}
\end{Highlighting}
\end{Shaded}

\begin{figure}
\includegraphics[width=0.75\linewidth]{202401311000-TiKZ_files/figure-latex/unnamed-chunk-59-1} \caption{cube}\label{fig:unnamed-chunk-59}
\end{figure}

\url{https://tex.stackexchange.com/questions/388621/optimizing-perspective-tikz-graphic}

\begin{figure}
\includegraphics[width=0.75\linewidth]{202401311000-TiKZ_files/figure-latex/unnamed-chunk-60-1} \caption{cube rotate}\label{fig:unnamed-chunk-60}
\end{figure}

\begin{figure}
\includegraphics[width=0.75\linewidth]{202401311000-TiKZ_files/figure-latex/unnamed-chunk-61-1} \caption{cube rotate}\label{fig:unnamed-chunk-61}
\end{figure}

\url{https://github.com/XiangyunHuang/bookdown-broken/blob/master/index.Rmd}

\begin{CJK}{UTF8}{bsmi}
\begin{figure}
\includegraphics[width=0.65\linewidth]{202401311000-TiKZ_files/figure-latex/skills-1} \caption{《现代统计图形》的技能栈}\label{fig:skills}
\end{figure}
\end{CJK}

\begin{figure}
\includegraphics[width=0.75\linewidth]{202401311000-TiKZ_files/figure-latex/unnamed-chunk-62-1} \caption{cube rotate}\label{fig:unnamed-chunk-62}
\end{figure}

\hypertarget{xy-pic-1}{%
\chapter*{xy-pic}\label{xy-pic-1}}
\addcontentsline{toc}{chapter}{xy-pic}

\url{https://bookdown.org/yihui/rmarkdown-cookbook/install-latex-pkgs.html}

\texttt{tinytex::install\_tinytex()}

the following xymatrix from LaTeX package xy for xy-pic is not shown or rendered in HTML:

\texttt{\$\textbackslash{}LaTeX\$} can only be used in HTML, not PDF

\xymatrix{U\ar[ddr]_{\psi}\ar[drr]^{\varphi}\ar[dr]|-{(x,y)}\\
 & X\times_{Z}Y\ar[d]^{q}\ar[r]_{p} & X\ar[d]_{f}\\
 & Y\ar[r]^{g} & Z
}

\[
\xymatrix{U\ar[ddr]_{\psi}\ar[drr]^{\varphi}\ar[dr]|-{(x,y)}\\
 & X\times_{Z}Y\ar[d]^{q}\ar[r]_{p} & X\ar[d]_{f}\\
 & Y\ar[r]^{g} & Z
}
\]

\hypertarget{rmarkdown}{%
\chapter{RMarkdown}\label{rmarkdown}}

\hypertarget{markdown}{%
\section{Markdown}\label{markdown}}

\hypertarget{bookdown}{%
\section{Bookdown}\label{bookdown}}

\hypertarget{references}{%
\chapter*{references}\label{references}}
\addcontentsline{toc}{chapter}{references}

\hypertarget{refs}{}

\begin{CJK}{UTF8}{bsmi}
\begin{CSLReferences}{0}{0}
\leavevmode\vadjust pre{\hypertarget{ref-R-bookdown}{}}%
\CSLLeftMargin{1. }%
\CSLRightInline{Xie, Y. \emph{\href{https://CRAN.R-project.org/package=bookdown}{Bookdown: Authoring Books and Technical Documents with r Markdown}}. (2023).}

\leavevmode\vadjust pre{\hypertarget{ref-xie2015}{}}%
\CSLLeftMargin{2. }%
\CSLRightInline{Xie, Y. \emph{\href{http://yihui.org/knitr/}{Dynamic Documents with {R} and Knitr}}. (Chapman; Hall/CRC, Boca Raton, Florida, 2015).}

\leavevmode\vadjust pre{\hypertarget{ref-noauthor_bookdown_2019}{}}%
\CSLLeftMargin{3. }%
\CSLRightInline{\href{https://community.rstudio.com/t/bookdown-books-on-the-web-downloading-and-converting-to-pdf/30268}{Bookdown books on the web: Downloading and converting to pdf - {R} {Markdown}}. \emph{Posit Community} (2019).}

\leavevmode\vadjust pre{\hypertarget{ref-ccjou2009}{}}%
\CSLLeftMargin{4. }%
\CSLRightInline{ccjou. \href{https://ccjou.wordpress.com/2009/10/21/\%e4\%ba\%8c\%e6\%ac\%a1\%e5\%9e\%8b\%e8\%88\%87\%e6\%ad\%a3\%e5\%ae\%9a\%e7\%9f\%a9\%e9\%99\%a3/}{二次型與正定矩陣}. (2009).}

\end{CSLReferences}
\end{CJK}

\end{document}
