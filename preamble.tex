\usepackage{booktabs}
% \usepackage{fontspec} %這個可能原本文檔就已經有了,放入時候check一下
% \usepackage{CJKutf8}
% \usepackage[UTF8]{inputenc}
\usepackage{CJK}
% \usepackage{xeCJK}

%英文字體調整(有時候交中文文件可能有規定對應的英文字體)
% \setmainfont{Times New Roman}
% \setmainfont{Noto Sans}

%中文字體main跟mono都需要哦,最後面的SC是簡體中文,也可以改成TC,不過SC的破字會比較少
% \setCJKmainfont{NotoSansTC-Regular.otf}
% \setCJKmonofont{NotoSansTC-Regular.otf}

% \usepackage{bm}
\usepackage{amsmath,amssymb}
% \usepackage[pagebackref=true]{hyperref}
% \usepackage[backref]{hyperref}
\usepackage[]{hyperref}
\usepackage[hyperpageref]{backref}
\hypersetup{
    colorlinks=true,
    linkcolor=blue,
    filecolor=magenta,      
    urlcolor=cyan
}

% \usepackage{fdsymbol} % vector over accent, but will use mathptmx
%% replace the rather ugly mathptmx \sum operator with the equivalent Computer Modern one
% \let\sum\relax
% \DeclareSymbolFont{CMlargesymbols}{OMX}{cmex}{m}{n}
% \DeclareMathSymbol{\sum}{\mathop}{CMlargesymbols}{"50}

\usepackage{cancel} % demo usepackage in PDF and require in HTML

% to wrap the text inside the margins of the PDF document when using code chunks in bookdown
\usepackage{fvextra}
\DefineVerbatimEnvironment{Highlighting}{Verbatim}{breaklines,commandchars=\\\{\}}

% \usepackage[backend=bibtex]{biblatex}
% \usepackage[backend=biber]{biblatex}
% \usepackage[]{biblatex}
% \DeclarePrintbibliographyDefaults{heading=bibintoc}

% \let\oldpb\printbibliography
% \renewcommand{\printbibliography}{\oldpb[heading=bibintoc]}

% SVG
\usepackage{svg}

% TikZ
\usepackage{tikz}
\usepackage{tikz-3dplot}
\usepackage{pgfplots}
\pgfplotsset{compat=1.15}
% animation
\usepackage[autoplay]{animate}

% xcolor colorbox
% https://www.overleaf.com/learn/latex/Using_colors_in_LaTeX
\usepackage[dvipsnames]{xcolor}
% \usepackage[svgnames]{xcolor}
% \usepackage[x11names]{xcolor}

\usepackage{mathrsfs}
\usetikzlibrary{arrows}
% \pagestyle{empty}
% \newcommand{\degre}{\ensuremath{^\circ}}

\usepackage[all]{xy}

% LaTeX Error: Too deeply nested
% https://stackoverflow.com/questions/57945414/too-deeply-nested-at-just-fourth-nesting-level-using-pandoc-with-markdown
\usepackage{enumitem}
\setlistdepth{20}
\renewlist{itemize}{itemize}{20}
\renewlist{enumerate}{enumerate}{20}
\setlist[itemize]{label=$\cdot$}
\setlist[itemize,1]{label=\textbullet}
\setlist[itemize,2]{label=--}
\setlist[itemize,3]{label=*}

% multicolumn
% https://bookdown.org/yihui/rmarkdown-cookbook/multi-column.html
\newenvironment{cols}[1][]{}{}

\newenvironment{col}[1]{\begin{minipage}{#1}\ignorespaces}{%
\end{minipage}
\ifhmode\unskip\fi
\aftergroup\useignorespacesandallpars}

\def\useignorespacesandallpars#1\ignorespaces\fi{%
#1\fi\ignorespacesandallpars}

\makeatletter
\def\ignorespacesandallpars{%
  \@ifnextchar\par
    {\expandafter\ignorespacesandallpars\@gobble}%
    {}%
}
\makeatother